\newpage
\section{Fourier- Reihen und Fourier- Transformationen}
\subsection{Definition: periodisch}
	$ f:\mr\rightarrow\mr $ heißt \textbf{periodisch} mit Periode $ w>0 $, falls gilt:\\
	$ f(t)=f(t+w) $ für alle $ t\in\mr $.\\
	($ f $ ist bestimmt durch ihre Werte auf $ [0,w[ $)\\
	Auch periodisch mit Periode $ 2w,3w,... $

\subsection{Beispiel}
	\subExBegin{a)}
	\item $ r>0 $. Dann $ \sin(rt),\cos(rt) $ periodisch mit Periode $ \frac{2\pi}{r} $.
	\item Ist $ f $ periodisch mit Periode $ w $, $ w'>0 $ gegeben, $ \overset{\sim}{f}(t):=f\lrr{\frac{wt}{w'}} $.\\
	$ w' $- periodisch
	\subExEnd
	
	Ziel: Approximation periodischer Funktion durch sogenannte trigonometrische Polynome.
	
\subsection{Definition: trigonometrische Polynome und Ordnung}
	Eine Funktion $ p:\mr\rightarrow\mr $ der Form $ p(t)=\frac{a_0}{2}+\limsum{k=1}{n}a_k\cdot\lrr{\frac{2\pi kt}{w}+b_k\cdot\sin\lrr{\frac{2\pi kt}{w}}} $ mit $ a_i,b_i\in\mr $, $ w>0 $, $ n\in\mn $ heißt \textbf{trigonometrisches Polynom der Ordnung n} (bezüglich Periode $ w $).\\
	Beachte: Terme unter Summenzeichen sind $ \frac{w}{k} $- periodisch.\\
	Also $ p(t) $ ist $ w $- periodisch (13.2a).
	$ \frac{w}{k} $ oszilliert immer schneller mit wachsendem $ k $.
	
	Aus $ p(t) $ lassen $ a_k,b_k $ wieder zurückgewinnen. Dazu:

\subsection{Satz: Orthogonalitätsrelation}
	$ m,n\in\mn $. Dann:
	\begin{enumerate}[a)]
	\item $ \limint{0}{2\pi}\sin(mt)\sin(nt)\intd{t}=\begin{cases}0&m\neq n\\\pi&m=n\end{cases} $
	\item $ \limint{0}{2\pi}\cos(mt)\cos(nt)\intd{t}=\begin{cases}0&m\neq n\\\pi&m=n\end{cases} $
	\item  $ \limint{0}{2\pi}\sin(nt)\cos(mt)\intd{t}=0 $
	\end{enumerate}
	\textbf{Beweis:} Additionstheoreme + Substitution
	\textbf{Bedeutung:} $ v= $ Raum der stetigen Funktionen auf $ [0,2\pi] $\\
	Skalarprodukt: $ (f|g):=\limint{0}{2\pi}f(t)g(t)\intd{t} $\\
	$ \lrc{\frac{1}{\sqrt{\pi}}\sin(mt),\frac{1}{\sqrt{\pi}}\cos(nt)} $ Orthonormalsystem
	
\subsection{Satz}
	Sei $ p(t)=\frac{a_0}{2}+\limsum{k=1}{n}\lrr{a_k\cos\lrr{\frac{2\pi kt}{w}}+b_k\sin\lrr{\frac{2\pi kt}{w}}} $ trigonometrisches Polynom.\\
	Dann ist $ a_k=\frac{2}{w}\limint{0}{w}p(t)\cos\lrr{\frac{2\pi kt}{w}}\intd{t} $\\
	für $ k=1,2,... $\\
	$ b_k=\frac{2}{w}\limint{0}{w}p(t)\sin\lrr{\frac{2\pi kt}{w}}\intd{t} $\\
	für $ k=1,2,... $
	
	\textbf{Beweis:} Nachrechnen: 13.4 + Substitution
	
	\textbf{Gegeben:} beliebige $ w $- periodische Funktion $ f $.\\
	\textbf{Ziel:} Approximation von $ f $ durch ein trigonometrisches Polynom.\\
	Wie wählt man $ a_k,b_k $?
	
	Idee: Ersetze in 2 Gleichungen in 13.5 $ p(t) $ durch $ f(t)\rightarrow a_k,b_k $.\\
	Unter gegebenen Voraussetzungen gute Approximation.
	
\subsection{Definition: Fourierreihe}
	\subExBegin{a)}
	\item Eine Reihe der Form
	\[\dfrac{a_0}{2}+\limsum{k=1}{\infty}\lrr{a_k\cos\lrr{\frac{2\pi kt}{w}}+b_k\sin\lrr{\dfrac{2\pi kt}{w}}}\]
	heißt (reelle) \textbf{Fourierreihe} zur Periode $ w $.\\
	Die $ a_i,b_j $ nennt man Fourierkoeffizient der Reihe
	\item Sei $ f $ eine $ w $- periodische Funktion die über $ [0,w] $ integrierbar ist (dann ist $ f $ über $ \mr $ lokal integrierbar).
	
	Die Reihe
	\[\dfrac{a_0}{2}+\limsum{k=1}{\infty}\lrr{a_k\cos\lrr{\frac{2\pi kt}{w}}+b_k\sin\lrr{\dfrac{2\pi kt}{w}}}\]
	mit
	\[a_k=\frac{2}{w}\limsum{0}{w}f(t)\cos\lrr{\dfrac{2\pi kt}{w}}\intd{t},\ k\in\mn_0\]
	\[b_k=\frac{2}{w}\limint{0}{w}f(t)\sin\lrr{\dfrac{2\pi kt}{w}}\intd{t},\ k\in\mn\]
	heißt die \textbf{Fourierreihe} zu $ f $.
	\subExEnd
	
\subsection{Satz}
	$ f:\mr\rightarrow\mr $ stetig, $ w $- periodisch.\\
	$ f $ sei auf $[0,w]$ stückweise stetig differenzierbar (d.h. $\exists 0=t_0<t_1<...<t_n=w$, so dass $f_{|_{[t_1,t_{i+1}]}}$ stetig differenzierbar in $i=,,,.n$).
	\begin{enumerate}[a)]
	\item Die Fourierreihe zu $f$ konvergiert gleichmäßig gegen $f$, d.h.
	\[f(t)=\dfrac{a_0}{2}+\limsum{k=1}{\infty}\lrr{a_k\cos\lrr{\dfrac{2\pi kt}{w}}+b_k\sin\lrr{\dfrac{2\pi kt}{w}}}\]
	wobei \gqm{gleichmäßig} bedeutet: \[\forall\epsilon>0\exists n\forall t\in\mr: \lrabs{f(t)-\lrr{\dfrac{a_0}{2}+\limsum{k=1}{n}a_k\cos\lrr{\dfrac{2\pi kt}{w}}+b_k\sin\lrr{\dfrac{2\pi kt}{w}}}}<\epsilon\]
	\item Ist $f$ eine gerade Funktion, d.h. $f(t)=t(-t)$ für alle $t\in\mr$, so ist $b_k=0$ für alle $k\in\mn$, d.h. $f(t)=\frac{a_0}{2}+\limsum{k=1}{\infty}a_k\cos\lrr{\frac{2\pi kt}{w}}$, reine \textbf{Cosinus- Reihe}.
	\item Ist $f$ eine ungerade Funktion, d.h. $f(t)=-f(-t)$ für alle $t\in\mr$, so ist $a_k=0$ für alle $k\in\mn_0$, d.h. $f(t)=\limsum{k=1}{\infty}b_k\sin\lrr{\frac{2\pi kt}{w}}$, reine \textbf{Sinus- Reihe}. (WHK 8.9 für stetig differenzierbare Funktionen).l
	\end{enumerate}
	
	In Anwendungen gibt es häufig unstetige Funktionen in $f$. Diese Funktionen lassen sich nicht gleichmäßig durch Polynome approximieren. Es gilt:
	
\subsection{Satz: \texorpdfstring{$f:\mr\rightarrow\mr$}{Funktion von R nach R} \texorpdfstring{$w$}{w}-periodisch und integrierbar auf \texorpdfstring{$[0,w]$}{[0,w]}}
	\[s_n(t)=\dfrac{a_0}{2}+\limsum{k=1}{n}a_k\cos\lrr{\dfrac{2\pi kt}{w}}+b_k\sin\lrr{\dfrac{2\pi kt}{w}}\]
	$ a_k,b_k $ wie in 13.6b
	
	Dann gilt $ \liminf{n}\limint{0}{w}\lrabs{f(t)s_n(t)}^2\intd{t}=0 $ (mittlere quadratische Abweichung).
	
\subsection{Bemerkung}
	Komplexe Fourierreihe
	\[\limsum{k=-\infty}{\infty}\alpha_k\cdot e^{i\frac{2\pi kt}{w}}\]
	$w$- periodisch, $\alpha\in\mc$
	
	Konvergenz bedeutet:
	$\limsum{k=0}{\infty}\alpha_k\cdot e^{i\frac{2\pi kt}{w}}$ konvergiert und\\
	$\limsum{k=0}{\infty}\alpha_k\cdot e^{i\frac{2\pi kt}{w}}$ konvergiert.
	
	$f:\mr\rightarrow\mc$ $w$- periodisch ($w\in\mr_{>0}$)\\
	komplex integrieren:\\
	(Real- und Imaginärteil integrieren: $\int f=\int R(f)+i\int I(f)$), so setze
	\[ \alpha_k=\dfrac{1}{w}\limint{0}{w}f(t)\cdot e^{-i\frac{2\pi kt}{w}}\intd{t} \]
	Ist $f$ stetig differenzierbar, so
	\[f(t)=\limsum{k=-\infty}{\infty}\alpha_k\cdot e^{i\frac{2\pi kt}{w}}\quad\mbox{(13.7)}\]
	($e^{i\frac{2\pi kt}{w}}=\cos\lrr{\frac{2\pi kt}{w}}+i\sin\lrr{\frac{2\pi kt}{w}}$)\\
	$f:\mr\rightarrow\mr$\\
	Imaginärteile fallen weg und wir erhalten 13.7 (WHK S.251/252).
	
\subsection{Fourier- Transformation}
	$w$- periodische Funktion $\longrightarrow$ Fourierreihe $\limsum{k=-\infty}{\infty}e^{i\frac{2\pi kt}{w}}$ (Periode $\frac{k}{w}$)
	\[\alpha_k=\dfrac{1}{w}\limint{0}{w}f(t)\cdot e^{-i\frac{2\pi kt}{w}}\intd{t}\qquad\mbox{Frequenz }\dfrac{k}{w}\]
	nicht periodische Funktion $f$ ($w\rightarrow\infty$) $\longrightarrow$ betrachte alle möglichen $s\in\mr$ als Frequenzen. ($\sum\rightarrow\int$)
	
	Man hätte gerne: $f(t)=\limint{-\infty}{\infty}g(n)\cdot e^{iut}\intd{u}$\\
	Wie findet man $g$?\\
	Durch die sogenannte Fouriertransformation von $f$:
	Dazu: $f:\mr\rightarrow\mc$ absolut integrierbar
	(d.h. $\limint{-\infty}{\infty}R(f),\ \limint{-\infty}{\infty},\ \limint{-\infty}{\infty}|f|$ existiert)
	\[f^{\land}(u):=\limint{-\infty}{\infty}f(t)e^{-iut}\intd{t}\]
	Fouriertransformation von $f$.
	\[f(t)=\dfrac{1}{2\pi}\limint{-\infty}{\infty}f^1(u)e^{iut}\intd{t}\]
	Inverse Fouriertransformation:
	\[f^{\lor}(t)=\dfrac{1}{2\pi}\limint{-\infty}{\infty}e^{iut}\intd{u}\]
	$ \lrr{f^\land}^\lor=f $
	
	Anwendung:\\
	Faltungsprodukt von $f,g$\\
	$ (f*g)(t)=\limint{-\infty}{\infty}f(s)g(t-s)\intd{s} $\\
	$ \underbrace{(f*g)(t)}_{\mbox{\scriptsize schwierig}}=\underbrace{\lrr{f^\land(t)\cdot g^\land(t)}^\lor}_{\mbox{\scriptsize leicht}} $
