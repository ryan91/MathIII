\chapter{Uneigentliche Integrale}

$a,b\in\mr, a<b$\\
$\limint{a}{b}f(x)\intd{x}=$Flächeninhalt

\begin{tikzpicture}[every plot/.append style={samples=50}]
	\draw [->] (-1,0) -- (5,0);
	\draw [->] (0,-1) -- (0,2.7);
	\draw plot [domain=0:5] (\x,{sqrt(\x)});
	\pattern [pattern=north west lines] (2,0) -- (2,1.41) -- plot [domain=2:4] (\x,{sqrt(\x)}) -- (4,0);
	\draw [thick] (2,.1) -- (2,-.1) node [anchor=north] {$a$};
	\draw [thick] (4,.1) -- (4,-.1) node [anchor=north] {$b$};
	%\pattern (0,0) -- (0,1) -- (1,1) -- (1,0);
	%\draw [curved] (0,0) -- (1,2) -- (4,3);
\end{tikzpicture}

bestimmtes Integral existiert für Regelfunktion (z.B. stetige Funktion)

$I$ bel. Intervall:\\
$f$ auf $I$ lokal integrierbar $\Leftrightarrow$ für alle $u,v\in I, u<v$ existiert $\limint{u}{v}f(x)\intd{x}$

$f$ stetig $\Rightarrow$ existiert Stammfunktion $F$ von $f$ (unbestimmtes Integral)\\
$\limint{u}{v}f(x)\intd{x}=F(v)-F(u), F$ diffbar $F'=f$

Frage: Kann man Flächeninhalt unter dem Graphen von $f$ für beliebige Intervalle definieren?

Geht nicht immer

\begin{tikzpicture}
	\draw [->] (-.5, 0) -- (2,0);
	\draw [->] (0,-.5) -- (0,4);
	\draw plot[domain=0:2] (\x,{\x*\x}) node[right] {$f(x)=x^2$};
	\pattern [pattern=north west lines] plot[domain=0:2] (\x,{\x*\x}) -- (2,0) -- (0,0);
\end{tikzpicture}

\gqm{undendlicher} Flächeninhalt

\section{Definition}
	\subExBegin{(a)}
		\item Sei $I=[a,b[$ ein halboffenes Intervall ($b=\infty$ zugelassen), $f:I\rightarrow\mr$ lokal integrierbar auf $I$ mit Stammfunktion $F$\\
			Setze $\limint{a}{b}f(x)\intd{x}=\limlim{t\rightarrow b}\limint{a}{t}f(x)\intd{x}=\limlim{t\rightarrow b}\lrr{F(t)-F(a)}$, falls der Grenzwert existiert.\\
			Dann $f$ \textbf{integrierbar} über $I$
			
			$\limint{a}{b}f(x)\intd{x}$ \textbf{(uneigentliches) Integral von $f$} über $I=[a,b[$\\
			Analog für $I=]a,b], b\in\mr$ ($a=-\infty$ zugelassen)\\
			$\limint{a}{b}f(x)\intd{x}=\limlim{t\rightarrow a}\limint{t}{b}f(x)\intd{x}=\limlim{t\rightarrow a}\lrr{F(b)-F(t)}$
		\item $I=]a,b[$ offenes Intervall ($a=-\infty, b=\infty$ zugelassen), $f,F$ wie in a)\\
			Wähle $x_0\in I$
			
			\begin{tikzpicture}[every plot/.append style={samples=40}]
				\draw (-3,0) -- (3,0);
				\draw [thick] plot [domain=-2.3:2.3] (\x,{(\x*\x)+0.2});
				\pattern [pattern=north west lines] (-2.5,0) -- (-2.5,6.25) -- plot [domain=-2.5:2.5] (\x,{\x*\x+0.2}) -- (2.5,0);
				\draw [thick] (-2,.1) -- (-2,-.1) node [anchor=north] {$a$};
				\draw [thick] (-.5,.1) -- (-.5,-.1) node [anchor=north] {$x_0$};
				\draw [thick] (2,.1) -- (2,-.1) node [anchor=north] {$b$};
			\end{tikzpicture}	
			
			$\limint{a}{b}f(x)\intd{x}=\limlim{t\rightarrow a}\limint{t}{x_0}f(x)\intd{x}+\limlim{t'\rightarrow b}\limint{x_0}{t'}f(x)\intd{x}$\\
			falls beide Grenzwerte existieren (Das ist dann unabhängig davon wie $x_0$ gwählt wird)
			
			\textbf{Beispiel}
			
			$a=-\infty, b=\infty$\\
			Definition ist stärker als folgende:\\
			$\limlim{t\rightarrow\infty}\limint{-t}{t}f(x)\intd{x}\quad\limlim{t\rightarrow\infty}\limint{-t}{t}\sin\lrr{x}\intd{x}=0$\\
			$x_0=0: \limint{0}{t}\sin\lrr{x}\intd{x}=\lra{-\cos\lrr{x}}_x^t=-\cos\lrr{t}+1$\\
			$\limlim{t\rightarrow\infty}\limint{0}{t}|sin\lrr{x}\intd{x}=1+\limlim{t\rightarrow\infty}\cos\lrr{t}$ existiert nicht.
			
			\begin{tikzpicture}[every plot/.append style={samples=40}]
				\draw plot [domain=-5:5] (\x,{sin(deg(\x)});
				\draw (-4.71,-1.3) node [anchor=north] {$-t$} -- (-4.71,1.3);
				\draw (-5,0) -- (5,0);
				\draw (0,-1.3) -- (0,1.3);
				\draw (4.71,-1.3) node [anchor=north] {$-t$} -- (4.71,1.3);
				\pattern [pattern=north west lines] (-1.5*pi,1) -- plot [domain=-1.5*pi:-1*pi] (\x,{sin(deg(\x))}) -- (-1.5*pi,0);
				\pattern [pattern=north west lines] (-1*pi,0) -- plot [domain=-1*pi:0] (\x,{sin(deg(\x))});
				\pattern [pattern=north west lines] (0,0) -- plot [domain=0:pi] (\x,{sin(deg(\x))});
				\pattern [pattern=north west lines] (pi,0) -- plot [domain=pi:1.5*pi] (\x,{sin(deg(\x))}) -- (1.5*pi,0);
			\end{tikzpicture}
	\end{enumerate}
	
\section{Beispiele}
	\begin{enumerate}[a)]
		\item $f(x)=\frac{1}{\sqrt{x}}, I=]0,1]$\\
			$0<t<1\quad \limint{t}{1}\frac{1}{\sqrt{x}}\intd{x}=\limint{t}{1}x^{-\frac{1}{2}}\intd{x}=\lra{2x^{\frac{1}{2}}}_t^1=2-2\sqrt{t}$\\
			$\limint{0}{1}\frac{1}{\sqrt{x}}=\limlim{t\rightarrow 0}\limint{t}{1}\frac{1}{\sqrt{x}}\intd{x}=\limlim{t\rightarrow 0}\lrr{2-2\sqrt{t}}=2$
		\item $f(x)=\frac{1}{x}, I=]0,1]$\\
			$0<t<1\quad \limint{t}{1}\frac{1}{x}\intd{x}=\lra{\ln\lrr{x}}_t^1=0-\ln\lrr{t}$\\
			$\limlim{t\rightarrow 0}\limint{t}{1}\frac{1}{x}\intd{x}=\limlim{t\rightarrow 0}\lrr{-\ln\lrr{t}}$ existiert nicht ($\infty$)
		\item $f(x)=\frac{1}{e^x}, I=[0,\infty[$\\
			$t>0\quad \limint{0}{t}\frac{1}{e^x}\intd{x}=\limint{0}{t}e^{-x}\intd{x}\ouset{=}{g(x)=e^x}{f(x)=-x}-\limint{0}{t}g\lrr{f\lrr{x}}f'\lrr{x}\intd{x}\ouset{=}{\smt{sub.}}{\smt{Regel}} -\limint{f(0)}{f(t)}g(u)\intd{u}=-\limint{0}{-t}e^u\intd{u}=\limint{-t}{0}e^u\intd{u}=\lra{e^u}_{-t}^0=1-e^{-t}=1-\frac{1}{e^t}$\\
			$\limint{0}{\infty}\frac{1}{e^x}\intd{x}=\limlim{t\rightarrow\infty}\limint{0}{t}\frac{1}{e^x}\intd{x}=\limlim{t\rightarrow\infty}\lrr{1-\frac{1}{e^t}}=1$
	\end{enumerate}
	
\section{Umkehrfunktionen der Winkelfunktionen}
	$\sin, \cos, \tan$ sind nicht bijektiv auf $\mr$\\
	$\sin$ ist streng monoton wachsend (also bijektiv) auf $\lra{-\frac{\pi}{2},\frac{\pi}{2}}$
	
	\begin{tikzpicture}[every plot/.append style={samples=40}]
		\draw (-2,0) -- (2,0);
		\draw (0,-1.2) -- (0,1.2);
		\draw plot [domain=-0.5*pi:0.5*pi] (\x,{sin(deg(\x))});
		\draw (-0.5*pi,.1) -- (-0.5*pi,-.1) node [anchor=north] {$-\frac{\pi}{2}$};
		\draw (0.5*pi,.1) -- (0.5*pi,-.1) node [anchor=north] {$\frac{\pi}{2}$};
	\end{tikzpicture}
	
	$\cos$ ist streng monoton wachsend auf $\lra{0,\pi}$\\
	$\tan=\frac{\sin}{\cos}$ ist streng monoton wachsend auf $\left]-\frac{\pi}{2},\frac{\pi}{2}\right[$
	
	\begin{tikzpicture}
		\draw (-2,0) -- (2,0);
		\draw (0,-3.5) -- (0,3.5);
		\draw [dashed] (-0.5*pi,-3) -- (-0.5*pi,3);
		\draw [dashed] (0.5*pi,-3) -- (0.5*pi,3);
		\draw plot [domain=-0.4*pi:0.4*pi] (\x,{tan(deg(\x))});
		\draw (-0.5*pi,.1) -- (-0.5*pi,-.1) node [anchor=north] {$-\frac{\pi}{2}$};
		\draw (0.5*pi,.1) -- (0.5*pi,-.1) node [anchor=north] {$\frac{\pi}{2}$};
	\end{tikzpicture}
	
	Auf diesen Intervallen kann man die Umkehrfunktion bilden:\\
	$\arcsin: \lra{-1,1}\rightarrow\lra{-\frac{\pi}{2},\frac{\pi}{2}}$\\
	$\arccos: \lra{-1,1}\rightarrow\lra{0,\pi}$\\
	$\arctan: \mr\rightarrow \left]-\frac{\pi}{2},\frac{\pi}{2}\right[$
	
	Ableitungen mit Hilfe der Regel $\lrr{f^{-1}}'\lrr{f\lrr{u}}=\frac{1}{f'\lrr{u}}$\\
	$\arcsin'\lrr{x}=\frac{1}{\sqrt{1-x^2}}$ auf $]-1,1[$\\
	$\arccos'\lrr{x}=-\frac{1}{\sqrt{1-x^2}}$ auf $]-1,1[$\\
	$\arctan'\lrr{x}=\frac{1}{1+x^2}$ auf $\mr$
	
	$\int\frac{1}{1+x^2}\intd{x}=\arctan\lrr{x}+c$\\
	$\int\frac{1}{\sqrt{1-x^2}}\intd{x}=\arcsin\lrr{x}+c$
	
\section{Beispiel}
	$f(x)=\frac{1}{1+x^2}$ auf $\mr$
	
	\begin{tikzpicture}
		\draw (-5,0) -- (5,0);
		\draw (0,0) -- (0,1.5);
		\draw plot [domain=-5:5,samples=50] (\x,{1/(1+\x*\x)});
		\pattern [pattern=north west lines] (-5,0) -- (-5,0.038) -- plot [domain=-5:5,samples=50] (\x,{(1/(1+\x*\x))}) -- (5,0);
		\draw (0.5,0.5) -- (1,1) node [anchor=south west] {$\scriptstyle f(x)$};
	\end{tikzpicture}
	
	$\limint{0}{t}\frac{1}{1+x^2}\intd{x} =\lra{\arctan\lrr{x}}_0^t=\arctan\lrr{t}$\\
	$\limlim{t\rightarrow\infty}\limint{0}{t}\frac{1}{1+x^2}\intd{x}=\limlim{t\rightarrow\infty}\arctan\lrr{t}=\frac{\pi}{2}$
	
	Aus Symmetriegründen\\
	$\limint{-\infty}{0}\frac{1}{1+x^2}\intd{x}=\frac{\pi}{2}$\\
	$\limint{-\infty}{\infty}\frac{1}{1+x^2}\intd{x}=\pi$
	
\section{Satz: Vergleichskriterium}
	$I=\lrg{a,b}$ beliebiges Intervall, $f,g: I\rightarrow\mr$\\
	Ist $\lrabs{f\lrr{x}}\leq g(x)$ für alle $x\in I$, $g$ sei über $I$ integrierbar, $f$ sei lokal integrierbar auf $I$.\\
	Dann sind $f$ und $\lrabs{f}$ über $I$ integrierbar und es gilt:\\
	$\lrabs{\limint{a}{b}f(x)\intd{x}}\leq\limint{a}{b}\lrabs{f(x)}\intd{x}\leq\limint{a}{b}g(x)\intd{x}$\\
	(Folgt aus Vergleichenden Kriteria für bestimmte Integrale auf abgeschlossenen Intervallen und Limes-Regeln
	
\section{Beispiel}
	\begin{enumerate}[a)]
		\item $f\lrr{x}=e^{-\frac{x^2}{2}}$ ist über $\mr$ lokal integrierbar (da stetig)\\
			\gqm{Gauß'sche Glockenkurve}
			
			Es gilbt keine Formel für $\int e^{-\frac{x^2}{2}}\intd{x}$\\
			$e^{-\frac{x^2}{2}}=\frac{1}{e^{\frac{x^2}{2}}}=\frac{1}{1+\frac{x^2}{2}+\frac{x^4}{4\cdot 2}+\dots}\leq\frac{1}{1+\frac{x^2}{2}}< 2\cdot\frac{1}{1+x^2}$
			
			11.4, 11.5:\\
			$e^{-\frac{x^2}{2}}$ ist integrierbar über $\mr$\\
			$\limint{-\infty}{\infty}e^{-\frac{x^2}{2}}\intd{x}<2\cdot\limint{-\infty}{\infty}\frac{1}{1+x^2}\intd{x}=2\pi$\\
			Man kann zeigen $\limint{-\infty}{\infty}e^{-\frac{x^2}{2}}\intd{x}=\sqrt{2\pi}$
		\item $\mu\in\mr, \sigma>0, f(x)=\frac{1}{\sqrt{2\pi}\cdot\sigma}e^{-\frac{\lrr{x-\mu}^2}{2\sigma^2}}$\\
			Dann gilt $\limint{-\infty}{\infty}f(x)\intd{x}=1$\\
			$f$ heißt Dichtefunktion der Normalverteilung mit Mittelwert $\mu$ und Standardabweichung $\sigma$.\\
			Für $\mu=0, \sigma=1$ ergibt sich die Standardnormalverteilung.
	\end{enumerate}
