\newpage
\section{Kurze Wiederholung und Erweiterung der linearen Algebra}
\subsection{Definition - $K$-Vektorraum}
	Sei $K$ ein Körper.\\
	Ein $K$-Vektorraum $V$ besitzt Verknüpfung $+$, bezüglich derer er eine kommutative Gruppe ist.\\
	(neutrales Element, $\sigma$ Nullvektor, Inverse zu $v:-v$)\\
	Außerdem existiert die Multiplikation von Vektoren (Elemente aus $V$) mit Skalaren (Elemente aus $K$), so dass gilt:\\
	$av\in V$ für alle $a\in K, v\in V$\\
	$\underbrace{a+b}_{\mbox{\scriptsize K}}v=\underbrace{av+bv}_{\mbox{\scriptsize V}}$ für alle $a,b\in K, v\in V$\\
	$a\lrr{v+w}=av+aw$ für alle $a\in K,v,w\in V$\\
	$\underbrace{a\cdot b}_{\mbox{\scriptsize K}}v=a\lrr{bv}$ für alle $a,b\in K, v\in V$\\
	$1v=v$ für alle $v\in V$ (1 ist neutrales Element in $K$ bezüglich Multiplikation)

\subsection{Wichtige Begriffe über Vektorräume}
	\subExBegin{a)}
		\item $K^n$ ist K- Vektorraum\\
		$\lrc{\lrv{a_1\\\vdots\\a_n}:a_i\in K}$, $\lrv{a_1\\\vdots\\a_n}+\lrv{b_1\\\vdots\\b_n}=
		\lrv{a_1+b_1\\\vdots\\a_n+b_n}$\\
		$a\lrv{a_1\\\vdots\\a_n}=\lrv{a\cdot a_i\\\vdots\\a\cdot a_n}$\\
		$\mz^3_3$\quad$\lrv{1\\2\\0}+\lrv{1\\2\\1}=\lrv{2\\1\\1}$\\
		$2\lrv{1\\2\\0}=\lrv{2\\1\\0}$\\
		$K[x]$ ist $K$- Vektorraum
		\item Unterräume
		\item Linearkombinationen\\
		$v_1,...,v_n\in V$, $a_1,...,a_n\in K$\\
		$\limsum{i=1}{n}a_iv_i$
		\item $M\mpo V$ $K$- Vektorraum\\
		$\lrg{M}_K=\lrc{\limsum{i=1}{n}a_iv_i:n\in\mn,a_i\in K,v_i\in M}$\\
		kleinster Unterraum von $V$, der $M$ enthält.

		Bsp.: $K=\mz_2$, $V=\mz^3_2$\\
		$M=\lrc{\lrv{0\\1\\0},\lrv{1\\1\\1}}$\\
		$\lrg{M}_{\mz_2}=\lrc{\lrv{0\\0\\0},\lrv{0\\1\\0},\lrv{1\\1\\1},\lrv{1\\0\\1}}$

		\item $v_1,...,v_n$ sind linear unabhängig, falls gilt:\\
		Wenn $a_1v_1+...+a_nv_n=\sigma$, dann $a_1=...=a_n=0$ (sonst linear abhängig)

		Bsp.: $\lrv{0\\1\\0},\lrv{1\\1\\1},\lrv{1\\0\\1}\in\mz_2^3$ sind linear abhängig:\\
		$\lrv{0\\1\\0}+\lrv{1\\1\\1}+\lrv{1\\0\\1}=\lrv{0\\0\\0}$\\
		$\lrv{0\\1\\0},\lrv{1\\1\\1},\lrv{1\\0\\1}\in\mz_3^3$\\
		$a\lrv{0\\1\\0}+b\lrv{1\\1\\1}+c\lrv{1\\0\\1}=\lrv{0\\0\\0}$, $a,b,c\in\mz$\\
		$b+c=0$\\
		$a+b=0$\\
		$b+c=0$ entfällt

		$c$ beliebig, $b=-c$, $a=c$\\
		$\lrv{0\\1\\0}+2\lrv{1\\1\\1}+\lrv{1\\0\\1}=\lrv{0\\0\\0}$

		\item $V$ endlich erzeugtes $K$- Vektorraum\\
		Basis: Linear unabhängiges Erzeugendensystem
	\subExEnd

\subsection{Wichtige Sätze}
	\subExBegin{a)}
		\item $V$ endlich erzeugter $K$- Vektorraum $V$ besitzt Basen und je zwei Basen haben die gleiche Anzahl von Vektoren: $\dim_K(V)$
		\item $V$ endlich dimensionaler Vektorraum. Jede linear unabhängige Menge von Vektoren in $V$ zu Basis von $V$ ergänzen (Basisergänzungs- Satz)
		\item $V$ endlich dimensionaler Vektorraum. Jedes endliche Erzeugendensystem von $V$ enthält Basis.
		\item $B=(v_1,...,v_n)$ geordnete Basis von $V$. $v\in V$. Dann existieren eindeutig bestimmte $a_1,...,a_n\in K$ mit $v=a_1v_1+...+a_nv_n$. $a_1,...,a_n$ \textbf{Koordinaten} von $v$ bezüglich $B$.

		Bsp.: $K^n$\\
		Kanonische basis $e_1,...,e_n$\\
		$e_j$ besitzt an $j$-ter Stelle eine $1$.\\
		$\lrv{a_n\\\vdots\\a_n}\in K^n=a_1e_1+...+a_ne_n$\\
		$a_i$ Koordinaten bezüglich kanonischer Basis.
	\subExEnd

\subsection{Matritzen}
	\subExBegin{a)}
		\item $m\times n$- Matrix $A$ über Körper $K$\\
		$(a_{ij})_{
			\begin{array}{l}
			i=1,...,m\\
			j=1,...,n
			\end{array}
		},\ a_{ij}\in K$
		\item $A,B$ $m\times n$- Matrix, $A+B=(a_{ij}+b_{ij})_{
			\begin{array}{l}
			i=1,...,m\\
			j=1,...,n
			\end{array}
		}$
		\item $A\cdot B=(c_{ij})$\quad $A$ $m\times l$- Matrix, $B$ $l\times n$- Matrix\\
		$c_{ij}=\limsum{k=0}{l}a_{il}b_{lj}$
		\item Zeilenring von $A$, $\mbox{Rang}(A)$: Maximalanzahl linear unabhängiger Zeilen von $A=\dim_K\lrg{z_1,...,z_m}$
	\subExEnd

\subsection{Wichtige Sätze über Matrizen}
	\subExBegin{a)}
		\item $m\times m$-Matrizen über $K$ bilden $K$-VR (Nullvektor ist Nullmatrix)
		\item $m\times m$-Matrizen über $K$ bilden Ring mit $1$ (1:$E_n$)\\
			Für $n\geq 2$ nicht kommutativ
		\item Elementare Zeilenumformungen ändert den (Zeilen-) Rang \textbf{nicht}.
		\item Transformationen einer Matrix durch elementaren Zeilenumformungen auf \\
			Zeilenstufenform. Dann Rang bestimmen.
		\item Gauß-Algorithmus zur Lösung von LGS $A\cdot x=b$\\
			Beziehungsweise um festzustellen dass keine Lösung existiert.
		\item $Ax=b$ lösbar $\Leftrightarrow \mbox{rg}\lrr{A}=\mbox{rg}(A,b)$\\
			Homogenes LGS $b=0$:\\
			$\dim\mathbb{L}=n-\mbox{rg}(A)$
	\subExEnd

\subsection{Definition - invertierbar}
	$A$ sei $n\times n$-Matrix über $K$.\\
	$A$ heißt \textbf{invertierbar}, falls $A$ in $M_n(K)$ bezüglich $\odot$ invertierbar ist.\\
	Das heißt, es existiert $A^{-1}\in M_n(K)$ mit $A\odot A^{-1}=A^{-1}\odot A=E_n$\\
	Zur Feststellung, ob $A$ invertierbar ist und wie man $A^{-1}$ berechnet später mehr.
