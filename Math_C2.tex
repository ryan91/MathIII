\section{Der Körper \texorpdfstring{$\mathbb{\mc}$}{C} der komplexen Zahlen}

Problem: In $\mr$ kann man keine Quadratzahlen aus negativen Zahlen ziegen.

Lösung: $\mr\subset\mc$

\subsection{Definition}
	Sei $\mc$ die Menge aller formalen Ausdrücke $a+bi$, wobei $a,b\in\mr$ und $i$ ein neues Symbol ist.\\
	$a+bi=a'+b'i\Leftrightarrow a=a'$ und $b=b'$\\
	Auf $\mc$ Addition $+$ und Multiplikation $\cdot$ definiert:\\
	$(a+bi)+(c+di)=(a+c)+(b+d)i$\\
	$(a+bi)(c+di)=(ac-bd)+(ad+bc)i$
	
	Schreibweise:\\
	Statt $a+0\cdot i$ schreibe $a$\\
	Statt $0+bi$ schreibe $bi$\\
	Statt $1i$ scheibe $i$.
	
	Dann: $i^2=(0+1i)(o+1i)=-1+0i=-1$
	
	Regel für die Multiplikation (Distributivgesetz):\\
	$ac+bci+adi+bdi^2=(ac-bd)+(bc+ad)i$
	
\subsection{Satz}
	\subExBegin{a)}
		\item Mit dem Verknüpfungen $+$ und $\cdot$ aus 2.1 wird $\mc$ ein Körper, der Körper der \textbf{komplexen Zahlen}.\\
		Nullelement: $0=0+0i$\\
		Einselement: $1=1+0i$
		\item Ist $a+bi\neq 0$, so ist $(a+bi)^{-1}=\frac{a}{a^2+b^2}-\frac{b}{a^2+b^2}i=$\\
		($c-di$ steht für $c+(-d)i$)\\
		Speziell: $b=0$ (d.h. $a\neq 0$)\\
		$(a+0i)^{-1}=\frac{1}{a}+0i$\\
		$a^{-1}=\frac{1}{a}$.\\
		(Inversenbild wie in $\mr$)
		
		$a=0$ ($b\neq 0$)\\
		$(bi)^{-1}=-\frac{1}{b}i$\\
		$i^{-1}=-i$
		\item $\lrc{a+0i:\ a\in\mr}$ entspricht bezüglich Addition und Multiplikation aus 2.1 dem Körper $\mr$ mit der üblichen Addition und Multiplikation in $\mr$ (Daher auch die Schreibweise $a$ statt $a+0i$).
	\subExEnd
	
	\textbf{Beweis}
	\subExBegin{a)}
		\item Nachrechnen
		\item Inverse bezüglich $\cdot$:\\
		$(a+bi)\cdot\lrr{\frac{a}{a^2+b^2}-\frac{b}{a^2+b^2}}\underset{\mbox{\scriptsize Def.}}{=}\lrr{\frac{a^2}{a^2+b^2}}+\lrr{\frac{ab}{a^2+b^2}-\frac{ab}{a^2+b^2}}i=1+0i=1$
		\item $(a+0i)+((c+0i))(a+c)+0i$\\
		$(a+0i)\cdot(c+0i)=ac+0i$
	\subExEnd
	
\subsection{Definition}
	Ist $z=a+bi\in\mc,\ a,b\in\mr$, so heißt $a$ \textbf{Realteil} von $z$ ($\mathcal{R}(z)$), $b$ heißt \textbf{Imaginärteil} von $z$ ($\mathcal{I}(z)$ (auch verwendet: $\mbox{Re}(z),\mbox{Im}(z)$).
	
	Die Zahl $\overline{z}=a-bi$ heißt die zu $z=a+bi$ \textbf{konjugiert komplexe Zahl}. Zahlen $z$ mit $\mathcal{R}(z)=0$ (d.h. $z=bi$) heißen auch \textbf{rein imaginär}.
	
\subsection{Geometrische Veranschaulichung von \texorpdfstring{$\mc$}{C}}
	$a+bi\leftrightarrow(a,b)\in\mr^2$\quad Übertragung der Operationen:\\
	$(a,b)+(c,d)=(a+c,b+d)$\\
	$(a,b)\cdot(c,d)=(ac-bd,ad+bc)$
	
	$0\leftrightarrow(0,0), 1\leftrightarrow(1,0)$\\
	$i\leftrightarrow(0,1)$
	
	% TODO insert diagram
	\begin{tikzpicture}
		\draw [-] (-2,0) -- (4,0);
		\draw [-] (0,-2) -- (0,4);
		\draw [-] (-.1,1) -- (.1,1) node [anchor=west] {$\scriptstyle i$};
		\draw [-] (1,.1) -- (1,-.1) node [anchor=north] {$\scriptstyle 1$};
		\draw [-] (.1,2) -- (-.1,2) node [anchor=east] {$\scriptstyle b$};
		\draw [-] (2,.1) -- (2,-.1) node [anchor=north] {$\scriptstyle a$};
		\draw (2,2) node [anchor=south west] {$z=a+bi$};
	\end{tikzpicture}
	
	$Z\rightarrow\overline{Z}$ entspricht Spiegelung an der horizontalen Achse.
	
	Horizontale Achse: Relle Zahlen\\
	Vertikale Achse: Rein imaginäre Zahlen
	
	Addition in $\mc$ entspricht der Vektoraddition
	
	Multiplikation von $a+bi$ mit $c\in\mr$\\
	$c\cdot(a+bi)=ca+cbi$
	
	\begin{tikzpicture}
	\draw [-] (-1,0) -- (4,0);
	\draw [-] (0,-1) -- (0,4);
	\draw [->] (0,0) -- (3,2) node [anchor=west] {$\scriptstyle (a,b)\leftrightarrow z$};
	\draw [->] (0,0) -- (4.5,3) node [anchor=west] {$\scriptstyle (a,cb)\leftrightarrow c\cdot z$};
	\end{tikzpicture}
	
	Multiplikation von $a+bi$ mit $i$:\\
	$i\cdot(a+bi)=-b+ai$
	
	\begin{tikzpicture}
	\draw [-] (-1,0) -- (4,0);
	\draw [-] (0,-1) -- (0,4);
	\draw [-] (0,0) -- (3,1) node [anchor=west] {$(a,b)$};
	\draw [-] (0,0) -- (-1,3) node [anchor=east] {$iz$};
	\draw [-] (3,.1) -- (3,-.1) node [anchor=north] {$a$};
	\draw [-] (-1,.1) -- (-1,-.1) node [anchor=north] {$-b$};
	\draw [-] (.1,1) -- (-.1,1) node [anchor=east] {$b$};
	\draw [-] (.1,2) -- (-.1,2) node [anchor=east] {$a$};
	\end{tikzpicture}
	
	Entspricht Drehung im $90^\circ$ gegen den Uhrzeigersinn.
	
	% TODO insert graphic
	
\subsection{Definition}
	\subExBegin{a)}
		\item \textbf{Absolutbetrag} von $z\in\mc$:\\
		$\lrabs{z}=+\sqrt{\mathcal{R}(z)^2+\mathcal{I}(z)^2}\in\mr$\\
		Ist $z\in\mr$, so entspricht das dem üblichen Absolutbetrag in $\mr$.
		
		Klar:\\
		$\lrabs{\mathcal{R}(z)},\lrabs{I(z)}\leq\lrabs{z}$
			
		\item Abstand von $z_1,z_2\in\mc$\\
		$d(z_1,z_2):=\lrabs{z_1-z_2}$
		
		\begin{tikzpicture}
		\draw [-] (-1,0) -- (4,0);
		\draw [-] (0,-1) -- (0,4);
		\draw (1,1) node [anchor=north west] {$\scriptstyle z_2$} -- (3,1) --  (3,2) node [anchor=west] {$\scriptstyle z_1$} -- (1,1);
		\end{tikzpicture}
	\subExEnd

\subsection{Satz}
	$z,z_1,z_2\in\mc$
	\subExBegin{a)}
		\item $\overline{(z_1+z_2)}=\overline{z_1}+\overline{z_2}$
		\item $\overline{z_1\cdot z_2}=\overline{z_1}\cdot\overline{z_2}$
		\item $\overline{\overline{z=z}}$
		\item $z=\overline{z}\ \Leftrightarrow\ z\in\mr$
		\item $\lrabs{z}=0\ \Leftrightarrow\ z=0$
		\item $\lrabs{z_1\cdot z_2}=\lrabs{z_1}\cdot\lrabs{z_2}$
		\item $\lrabs{z_1+z_2}\leq\lrabs{z_1}+\lrabs{z_2}$ (Dreiecksregel)
		\item $z\cdot\overline{z}=\lrabs{z}^2$\\
		Für $z\neq 0$ ist $z^{-1}=\frac{1}{\lrabs{z}^2}\cdot\overline{z}$
	\subExEnd