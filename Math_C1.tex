\newpage
\section{Algebraische Strukturen}
\subsection{Definiton - Verknüpfung}
	Sei $X\neq\mvoid$ eine Menge.\\
	Eine \textbf{Verknüfpung} auf $X$ ist Abbildung $*:\begin{cases}X\times X \rightarrow X\\\underbrace{\lrr{a,b}}_{a,b\in X}\mapsto a*b\in X\end{cases}$\\
	Man spricht vom 'Produkt' von $a$ und $b$.
\subsection{Beispiele}
	\subExBegin{a)}
		\item Die Addition und Multiplikation sind Verknüpfungen auf $\mn,\mz,\mq,\mr$.
		\item Die Division ist \textbf{keine} Verknüpfung auf $\mn$. \\
			Die Division ist eine Verknüpfung auf $\mq\mnot\lrc{0}$
		\item $X=\mathcal{P}\lrr{M}=$ Menge aller Teilmengen von $M$.\\
			$\mand,\mor,\mnot$ sind Verknüpfungen auf $X$.
		\item $M\mnot\mvoid$ Menge, $X=$ Menge aller Abbildungen $M\rightarrow M$ \\
			Die Hintereinanderausführung (Verkettung) $\circ$ ist eine Verknüpfung. \\
			$f:M\rightarrow M \quad g:M\rightarrow M$\\
			$f\circ g:\begin{cases}M\rightarrow M\\m\mapsto f\lrr{g(m)}\end{cases}$
		\item $n\in\mn\quad n>1$\\
			$\mz_n=\lrc{0,1,\dots,n-1}$\\
			Definiere Verknüpfungen $\oplus$ und $\odot$ auf $\mz_n$:\\
			$a\oplus b:=\lrr{a+b}\mod n\in\mz_n$\\
			$a\odot b:=\lrr{a\cdot b}\mod n\in\mz_n$\\
			$n=17$\\
			$5\oplus 8=\lrr{5+8}\mod 17=13$\\
			$5\oplus 14=19\mod 17=2$\\
			$5\odot 14=70\mod 17=2$
\newpage
		\item $X=\lrr{a,b}$\\
			Definiere die Verknüpfung\\
			$\begin{array}{c|cc}
				l.\backslash r.&a&b\\\hline
				a&b&b\\
				b&a&a
			\end{array}$
			
			$a*b=b\quad b*a=a$\\
			$\lrr{a*a}*a=b*a=a$\\
			$a*\lrr{a*a}=a*b=b$
		\item $M\neq\mvoid$ Menge und $X$ die Menge aller endlichen Folgen ('Wörter') mit Elementen aus $M$.\\
			$M=\lrc{0,1}\qquad\lrr{0,1,1,0,1}\leftrightarrow 01101$\\
			Verknüpfung auf $X$:\\
			Hintereinanderschreiben (Konkatenation)
		\item $M=\lrc{0,1}$\\
			Aussagenlogische Junktoren $\land,\lor,\Rightarrow,\Leftrightarrow,\lxor,\dots$ liefern Verknüpfungen auf $M$.
		\item $X=M_n\lrr{\mr}=$ Menge der $nxn$-Matrizen über $\mr$.\\
			Die atrizenaddition und -multiplikation sind beides Verknüpfungen auf $X$.
	\subExEnd
\subsection{Definition - Halbgruppe, Monoid, Gruppe}
	Sei $x\neq\mvoid$ eine Menge mit Verknüpfung $*$.
	\subExBegin{a)}
		\item $X$, genauer $\lrr{X,*}$, heißt \textbf{Halbgruppe}, falls gilt: \\
			$a*\lrr{b*c}=\lrr{a*b}*c$ für alle $a,b,c\in X$ (Assoziativgesetz)
		\item $\lrr{X,*}$ heißt \textbf{Monoid}, falls $\lrr{X,*}$ eine Halbgruppe ist und ein $e\in X$ existiert mit $e*x=x*e=x$ für alle $x\in X$. \\
			$e$ heißt \textbf{neutrales Element} bezüglich *.
		\item Der Monoid $\lrr{X,*}$ heißt \textbf{Gruppe}, falls es zu jedem $x\in X$ ein $y\in X$ gibt (abhängig von $x$) mit $x*y=y*x = e$.\\
			$y$ heißt \textbf{inverses Element} zu $x$.
		\item Halbgruppe/Monoid/Gruppe $\lrr{X,*}$ heißt \textbf{kommutativ} (bei Gruppen \textbf{abelsch}), falls gilt:\\
			$a*b=b*a$ für alle $a,b\in X$ \\
			(N.H.Abel, 1802-1829)
	\subExEnd
\subsection{Bemerkung}
	In Halbgruppen liefert jede sinnvolle Klammerung eines Produktes von $n$ Elementen $\lrr{n\geq 3}$ dasselbe Element (an der Reihenfolge der Faktoren nichts ändern!).\\
	$n=4$\\
	$\underbrace{\lrr{a*\lrr{b*c}}}*d\underset{\mbox{\scriptsize Ass.}}{=}\lrr{\underbrace{\lrr{a*b}}*c}*d\underset{\mbox{\scriptsize Ass.}}{=}\underbrace{\lrr{a*b}}*\underbrace{\lrr{c*d}}\underset{\mbox{\scriptsize Ass.}}{=}a*\lrr{b*\lrr{c*d}}\underset{\mbox{\scriptsize Ass.}}{=}a*\lrr{\lrr{b*c}*d}$\\
	Deshalb lässt man in Halbgruppe bei Produkten i.d.R. Klammern weg.
	
	\subsection{Proposition}
	
	\subExBegin{a)}
		\item Im einem Monoid $(X,*)$ ist das neutrale Element eindeutig bestimmt.
		\item Ist $(X,*)$ ein Monoid und besitzt $a\in X$ ein iverses Element in $X$, dann ist dieses inverse Element eindeutig bestimmt. Schreibe: $a^{-1}$
		\item Ist $(X,*)$ ein Monoid und besitzten $a,b\in X$ inverse Elemente in $X$, dann besitzt auch $a*b\in X$ ein inverses Element in $X$:\\
		$(a*b)^{-1}=b^{-1}*a^{-1}$
		\item Die Menge der Elemente eines Monoids $(X,*)$, welche ein Inverses besitzen, bilden bezüglich $*$ eine Gruppe.
	\subExEnd
	
	Bsp. für d): $(\mz,\cdot)$ Elemente mit Inversem in $\mz$\\
	$1,-1$\quad ($\lrc{1,-1,\cdot}$ Gruppe)
	
	\textbf{Beweis:}
	\subExBegin{a)}
		\item Angenommen, $e_1,e_2$ sind neutrale Elemente in $(X,*)$.$e_1\underset{e_2\mbox{\scriptsize\ neut. El.}}{=}e_1*e_2\underset{e_1\mbox{\scriptsize\ neut. El.}}=e_2$
		\item Angenommen, $a$ besitzt zwei inverse Elemente $b_1, b_2$. Dann $a*b_1=e$ und $b_2*a=e$\\
		$b_1=e*b_1=(b_2*a)*b_1=b_2*(a*b_1)=b_2*e=b_2$
		\item $(a*b)*(b^{-1}*a^{-1})=a*(b*b^{-1})*a^{-1}=(a*e)*a^{-1}=a*a^{-1}=e$\\
		Analog: $(b^{-1}*a^{-1})*(a*b)=e$. Behauptung folgt.
		\item $X*=\lrc{a\in X: a\mbox{ besitzt inverses in }X}$\\
		$(X^*,*)$ Gruppe:
		\subExBegin{1)}
			\item $*$ ist Verknüpfung auf $X^*$. D.h. $a,b\in X^*\Rightarrow a*b\in X^*$, folgt aus c)
			\item Das Assoziativgesetz gilt: in $X*$, da es in $X gilt$.
			\item $e\in X$ liegt auch in $X^*$, da $e*e=e$, d.h. $e^{-1}=e$.
			\item $a\in X^*$, so existiert $a^{-1}\in X$. Tatsächlich: $a^{-1}\in X^*$, dann $\lrr{a^{-1}}^{-1}=a$\\
			$a^{-1}*a=a*a^{-1}=r$
		\subExEnd
	\subExEnd
	
	\subsection{Beispiel}
	
	\subExBegin{a)}
		\item $\mn,\mz,\mq,\mr$ sind Halbgruppe bezüglich der Addition. $\mn_0,\mz,\mq,\mr$ sind Monoide bezüglich Addition (neutrales Element $0$).\\
		$\mz,\mq,\mr$ sind Gruppen bezüglich $+$ (inverses element zu $a$ ist $-a$).
		\item $\mn,\mz,\mq,\mr$ sind Monoide bezüglich Multiplikation (neutrales Element $1$). Keines der $\mn,\mz,\mq,\mr$ ist Gruppe bezüglich der Multiplikation (In $\mz,\mq,\mr$, da $0$ kein Inverses Element besitzt). $\mq\mnot\lrc{0},\mr\mnot\lrc{0}$ Gruppe bezüglich der Multiplikation.
		\item Division: Ist Verknüpfung auf $\mq\mnot\lrc{0}$ oder $\mr\mnot\lrc{0}$. Ist keine Halbgruppe.\\
		$(2:3):4=\frac{2}{3}\cdot\frac{1}{4}\cdot\frac{1}{6}$\\
		$2:(3:4)=2\cdot\frac{4}{3}=\frac{8}{3}$
		\item $X=\mathcal{P}(M)$ mit $\mor$ oder $\mand$ ist Monoid:\\
		Assoziativgesetz: Gilt\\
		$\mvoid$ neutrales Element bezüglich $\mor$. $M$ ist neutrales Element bezüglich $\mand$. Falls $M\neq\mvoid$, ist $\mathcal{P}(M)$ keine Halbgruppe bezüglich $\mnot$.\\
		Wähle $a\in M$.\\
		$(\lrc{a}\mnot\mvoid)\mnot\lrc{a}=\mvoid$\\
		$\lrc{a}\mnot(\mvoid\mnot\lrc{a})=\lrc{a}\neq\mvoid$
		\item Menge aller Abblidungen $M\rightarrow M$ mit Hintereinanderausführungen als Verknüpfung:\\
		Monoid, neutrales Element $id_M$. Genau der bijektiven Abbildungen $M\rightarrow M$ besitzen ein Inverses, die Umkehrabbildung $\alpha^{-1}$. $\alpha\circ\alpha^{-1}=\alpha^{-1}\circ\alpha=id_M$.\\
		Nach 1.5d): Bijektive Abbildung $M\rightarrow M$ bilden bezüglich Hintereinanderausführungen eine Gruppe.
		\item Menge aller endlichen Folgen über Alphabet $A$ mit Konkatenation als Verknüpfung.\\
		Leere Folge (leeres Wort): Neutrales Element. Keine nichtleere Folge hat ein inverses Element.
		\item $n\times n$- Matritzen über $\mr$: $M_n(\mr)$\\
		$(M_n(\mr),+)$ ist abelsche Gruppe. Neutrales Element: Nullmatrix\\
		Inverses zu $A:-A$.\\
		$(M_n(\mr),\underset{\mbox{\scriptsize Matr.mult.}}{\cdot})$ Monoid, neutrales Element\\
		Einheitsmatrix $E_n=\begin{pmatrix}1&\dots&0\\\vdots&1&\vdots\\0&\dots&1
		\end{pmatrix}$\\
		Keine Gruppe. Nullmatrix besitzt kein Inverses Element bezüglich der Multiplikation.\\
		$0\cdot A=0$
		\item $(\mz,\oplus)$\\
		$a\oplus b=(a+b)\mod n$ Gruppe\\
		$(a\oplus b)\oplus c=(((a+b)\mod n)+c)\mod n=((a+b)+c)\mod n$\\
		$a\oplus(b\oplus c)=(a+(b+c)\mod n))\mod n=(a+(b+c))\mod n$\\
		neutrales Element zu $0$: $0$\\
		Inverses zu $a\in\lrc{1,...,n-1}$ ist $-a\mod n=n-a$\\
		$a\oplus(n-a)=(a+n-a)\mod n=n\mod n=0$
		\item $(\mz_n,\odot)$ ist für jedes $n>1$ Monoid.\\
		$a\odot b=a\cdot b\mod n$\\
		neutrales Element: $1$\\
		Keine Gruppe: $0$ ist nicht invers.\\
		$\mz_4=\lrc{0,1,2,3}$\quad $2$ ist nicht invers bezüglich $\odot$.\\
		$2\odot 0=0$\\
		$2\odot 1=2$\\
		$2\odot 2=0$\\
		$2\odot 3=2$\\
		$2$ ist nicht invers bezüglich $\odot$.
		
		$3\odot 3=3\cdot 3\mod 4=1$\\
		$3$ hat ein inverses Element bezüglich $0$.
	\subExEnd
	
	\subsection{Satz}
	
	$n\in\mn,n>1$.
	\subExBegin{a)}
		\item Die Elemente in $\mz_n$, die bezüglich $\odot$ ein Inverses besitzen, sind genau die $a\in Z_n$ mit $ggT(a,n)=1$. Für solche $a$ berechnet man das Inverse $a_{-1}$ bezüglich $\odot$ folgendermaßen:\\
		Bestimme mit dem erweiterten Euklidischen Algorithmus $s,t\in mz$ mit $s\cdot a+t\cdot m=1(=ggT(a,n))$. Dann $a^{-1}=s\mod n$
		
		\item $Z_n^*:=\lrc{a\in\mz_n:\ggT(a,n)=1}$ ist dann Gruppe bezüflich $\odot$.\\
		$\lrabs{\mz_n^*}=\lrabs{\lrc{0\leq a\leq n:\ggT(a,n)=1}}=\phi(n)$ \textbf{Euler'sche} $\mathbf{\phi}$\textbf{- Funktion}\\ nach Leonard Euler, 1707,1783, Basel, St. Petersburg)
		
		\item Ist $p$ Primzahl, so ist $\mz_p^*=\mz_p\mnot\lrc{0}$ Gruppe bezüglich $\odot$.
	\subExEnd
	
	\textbf{Beweis:}
	\subExBegin{a)}
		\item Angenommen, $a\in\mz_n$ hat bezüglich $\odot$ Inverses $b\in\mz_n$. Das heißt, $a\odot b=1$. Also $a\cdot b\mod n=1$, das heißt $a\cdot b=k\cdot n+1$ für ein $jk\in\mz$. Sei $d$ ein Teiler von $a$ und von $n$. Da $d|a\Rightarrow d|a\cdot b$, $d|k\cdot n\Rightarrow d|a\cdot b-k\cdot n=1$, d.h. $d=\pm 1$. Also ist $\ggT(a,n)=1$\\
		Umgekehrt: Sei $a\in\mz_n$ und $\ggT(a,n)=1$. Bestimme mit dem erweiterten Euklidischen Algorithmus $s,t\in\mz:s\cdot a+t\cdot n=1\Leftrightarrow 1=(s\cdot a+t\cdot n)\mod n=((s\mod n)\cdot(a\mod n)+(t\mod n)\cdot(n\mod n))\mod n=((s\mod n)\cdot a)\mod n=(s\mod n)\odot a$\\
		Also $a^{-1}=s\mod n$
		
		\item Folgt aus 1.5 d) und Teil a)
		\item Folgt aus b)
	\subExEnd
	
	\subsection{Beispiel}
	
	$n=8$, $a=5$\\
	$\ggT(8,5)=1$. $a$ hat Inverses in $(\mz_8,\odot)$.
	Erweiterter Euklidischer Algorithmus: $1=(-3)\cdot 5+2\cdot 8$\\
	$5^{-1}=(-3)\mod 8=5$\\
	$5\oplus 5=5\cdot 5\mod 8=1$
	
	\subsection{Beispiel}
	
	Sei $M=\lrc{1,...,n}$. Die Menge der bijektiven Abbildungen $N\rightarrow M$ (Permutationen) bilden nach 1.6e) bezüglich der Hintereinanderausführung eine Gruppe. Diese Gruppe wird \textbf{symmetrische Gruppe vom Grad $n$} genannt, $S_n$.\\
	$\lrabs{S_n}=n!=1\cdot 2\cdot 3\cdot\dots\cdot n$.\\
	$n=3$. $\pi=\begin{pmatrix}1&2&3\\3&2&1\end{pmatrix}\in S_3,\pi^{-1}=\begin{pmatrix}1&2&3\\3&2&1\end{pmatrix}=\pi$
	
	$\delta=\begin{pmatrix}1&2&3\\2&3&1\end{pmatrix}
	\quad\delta^{-1}=
	\begin{pmatrix}1&2&3\\3&1&2\end{pmatrix}$\\
	$\delta^{-1}\circ\delta=
	\begin{pmatrix}1&23\\1&2&3\end{pmatrix}=\mbox{id}$\\
	$\pi\circ\delta=\begin{pmatrix}1&2&3\\2&1&3\end{pmatrix},\delta\circ\pi=\begin{pmatrix}1&2&3\\1&3&2\end{pmatrix}$\\
	$\pi\circ\delta\neq\delta\circ\pi$\\
	$S_3$ ist nicht abelsch.
	
	Ab jetzt: Beliebige Gruppe $(G,\cdot)$\\
	Statt $a\cdot b$ schreibt man $ab$.
	
	\subsection{Gleichunglösen in Gruppen}
	
	Sei $(G,\cdot)$ Gruppe, $a,b\in G$.
	\subExBegin{a)}
		\item Es gibt genau ein $x\in G$ mit $a\cdot x=b$, nämlich $x$, nämlich $x=a^{-1}\cdot b$ (\gqm{Teilen} von links = Multiplizieren von links mit $a^{-1}$).
		
		\item Es gibt genau ein $y\in G$ mit $y\cdot a=b$, nämlich $y=b\cdot a^{-1}$ (\gqm{Teilen} von rechts = Multiplizieren von rechs mit $a^{-1}$)
		
		\item Ist $a\cdot x=b\cdot x$ für ein $x\in G$, so ist $a=b$. Ist $y\cdot a=y\cdot b$ für ein $y\in G$, so ist $a=b$.
	\subExEnd
	
	\textbf{Beweis:}
	\subExBegin{a)}
		\item $x=a^{-1}\cdot b$ ist Lösung\\
		$a\cdot(a^{-1}\cdot b)=(a\cdot a^{-1})\cdot b=e\cdot b=b$\\
		Eindeutigkeit: $a\cdot x=b\Rightarrow x=e\cdot x=(a^{-1}\cdot a)\cdot x=a^{-1}(a\cdot x)=a^{-1}\cdot b$
		
		\item Analog zu a)
		\item $a\cdot x=b\cdot x$\\
		Multipliziere beide Seiten con rechts mit $x^{-1}$. $a=(a\cdot x)\cdot x^{-1}=(b\cdot x)\cdot x^{-1}=b$
	\subExEnd
	
	\subsection{Definition}
	
	Sei $(a,\cdot)$ eine Gruppe, $\mvoid\neq u\mpo G$. $u$ heißt \textbf{Untergruppe} von $G$, falls $u$ bezüglich $\cdot$ selbst eine Gruppe ist.\\
	Bezeichnung: $U\leq G$
	
	\subsection{Bemerkung}
	
	\subExBegin{a)}
		\item Man kann zeigen: Neutrales Element von $U\leq G$ = neutrales Element von $G$. Daher auch: Inverse von Elementen in $u$ = Inverse in $G$.
		\item Um zu zeigen, dass $u\neq\mvoid$ eine Untergruppe von $G$ ist, genügt es, zu zeigen:
		\subExBegin{(1)}
			\item $\forall u,v\in U:u\cdot v\in U$
			\item $\forall u\in U: u^{-1}\in U$
		\subExEnd
		Denn Assoziativgesetz gilt in $U$, da es in $G$ gilt.\\
		Neutrales Element: $\exists u\in U$, da $U\neq\mvoid$. Nach (2): $u^{-1}\in U$. Nach (1): $e=u\cdot u^{-1}\in U$
	\subExEnd
	
	\subsection{Satz}
	
	Sei $G$ eine endliche Gruppe
	\subExBegin{a)}
		\item \textbf{Satz von Lagrange}\\
		Ist $U\leq G$, so ist $\lrabs{U}\big|\lrabs{G}$\\
		($\lrabs{G}$ = \textbf{Ordnung} von $G$)
		
		\item Ist $g\in G$, so ist, $g^{\lrabs{G}}=e$\\
		Die kleinste natürliche Zahl $n$ (abhängig von $g$) mit $g^n=e$ heißt \textbf{Ordnung} von $g$.\\
		Man kann $\lrc{g=g^1,g^2,g^3,...,g^n=e}\leq G$ zeigen.\\
		$\lrabs{\lrc{g,g^2,...,g^n}}=n$.
		
		\textbf{Beweis:} WHK Satz 4.15, 4.22
	\subExEnd