\chapter{Algebraische Strukturen}
\section{Definiton: Verknüpfung}
	Sei $X\neq\mvoid$ eine Menge.\\
	Eine \textbf{Verknüfpung} auf $X$ ist Abbildung $*:\begin{cases}X\times X \rightarrow X\\\underbrace{\lrr{a,b}}_{a,b\in X}\mapsto a*b\in X\end{cases}$\\
	Man spricht vom 'Produkt' von $a$ und $b$.
\section{Beispiele}
	\subExBegin{a)}
		\item Die Addition und Multiplikation sind Verknüpfungen auf $\mn,\mz,\mq,\mr$.
		\item Die Division ist \textbf{keine} Verknüpfung auf $\mn$. \\
			Die Division ist eine Verknüpfung auf $\mq\mnot\lrc{0}$
		\item $X=\mathcal{P}\lrr{M}=$ Menge aller Teilmengen von $M$.\\
			$\mand,\mor,\mnot$ sind Verknüpfungen auf $X$.
		\item $M\mnot\mvoid$ Menge, $X=$ Menge aller Abbildungen $M\rightarrow M$ \\
			Die Hintereinanderausführung (Verkettung) $\circ$ ist eine Verknüpfung. \\
			$f:M\rightarrow M \quad g:M\rightarrow M$\\
			$f\circ g:\begin{cases}M\rightarrow M\\m\mapsto f\lrr{g(m)}\end{cases}$
		\item $n\in\mn\quad n>1$\\
			$\mz_n=\lrc{0,1,\dots,n-1}$\\
			Definiere Verknüpfungen $\oplus$ und $\odot$ auf $\mz_n$:\\
			$a\oplus b:=\lrr{a+b}\mod n\in\mz_n$\\
			$a\odot b:=\lrr{a\cdot b}\mod n\in\mz_n$\\
			$n=17$\\
			$5\oplus 8=\lrr{5+8}\mod 17=13$\\
			$5\oplus 14=19\mod 17=2$\\
			$5\odot 14=70\mod 17=2$
\newpage
		\item $X=\lrr{a,b}$\\
			Definiere die Verknüpfung\\
			$\begin{array}{c|cc}
				l.\backslash r.&a&b\\\hline
				a&b&b\\
				b&a&a
			\end{array}$

			$a*b=b\quad b*a=a$\\
			$\lrr{a*a}*a=b*a=a$\\
			$a*\lrr{a*a}=a*b=b$
		\item $M\neq\mvoid$ Menge und $X$ die Menge aller endlichen Folgen ('Wörter') mit Elementen aus $M$.\\
			$M=\lrc{0,1}\qquad\lrr{0,1,1,0,1}\leftrightarrow 01101$\\
			Verknüpfung auf $X$:\\
			Hintereinanderschreiben (Konkatenation)
		\item $M=\lrc{0,1}$\\
			Aussagenlogische Junktoren $\land,\lor,\Rightarrow,\Leftrightarrow,\lxor,\dots$ liefern Verknüpfungen auf $M$.
		\item $X=M_n\lrr{\mr}=$ Menge der $n\times n$-Matrizen über $\mr$.\\
			Die atrizenaddition und -multiplikation sind beides Verknüpfungen auf $X$.
	\subExEnd
\section{Definition: Halbgruppe, Monoid, Gruppe}
	Sei $x\neq\mvoid$ eine Menge mit Verknüpfung $*$.
	\subExBegin{a)}
		\item $X$, genauer $\lrr{X,*}$, heißt \textbf{Halbgruppe}, falls gilt: \\
			$a*\lrr{b*c}=\lrr{a*b}*c$ für alle $a,b,c\in X$ (Assoziativgesetz)
		\item $\lrr{X,*}$ heißt \textbf{Monoid}, falls $\lrr{X,*}$ eine Halbgruppe ist und ein $e\in X$ existiert mit $e*x=x*e=x$ für alle $x\in X$. \\
			$e$ heißt \textbf{neutrales Element} bezüglich *.
		\item Der Monoid $\lrr{X,*}$ heißt \textbf{Gruppe}, falls es zu jedem $x\in X$ ein $y\in X$ gibt (abhängig von $x$) mit $x*y=y*x = e$.\\
			$y$ heißt \textbf{inverses Element} zu $x$.
		\item Halbgruppe/Monoid/Gruppe $\lrr{X,*}$ heißt \textbf{kommutativ} (bei Gruppen \textbf{abelsch}), falls gilt:\\
			$a*b=b*a$ für alle $a,b\in X$ \\
			(N.H.Abel, 1802-1829)
	\subExEnd
\section{Bemerkung: Klammerung in Halbgruppen}
	In Halbgruppen liefert jede sinnvolle Klammerung eines Produktes von $n$ Elementen $\lrr{n\geq 3}$ dasselbe Element (an der Reihenfolge der Faktoren nichts ändern!).\\
	$n=4$\\
	$\underbrace{\lrr{a*\lrr{b*c}}}*d\underset{\mbox{\scriptsize Ass.}}{=}\lrr{\underbrace{\lrr{a*b}}*c}*d\underset{\mbox{\scriptsize Ass.}}{=}\underbrace{\lrr{a*b}}*\underbrace{\lrr{c*d}}\underset{\mbox{\scriptsize Ass.}}{=}a*\lrr{b*\lrr{c*d}}\underset{\mbox{\scriptsize Ass.}}{=}a*\lrr{\lrr{b*c}*d}$\\
	Deshalb lässt man in Halbgruppe bei Produkten i.d.R. Klammern weg.

	\section{Proposition}

	\subExBegin{a)}
		\item Im einem Monoid $(X,*)$ ist das neutrale Element eindeutig bestimmt.
		\item Ist $(X,*)$ ein Monoid und besitzt $a\in X$ ein iverses Element in $X$, dann ist dieses inverse Element eindeutig bestimmt. Schreibe: $a^{-1}$
		\item Ist $(X,*)$ ein Monoid und besitzten $a,b\in X$ inverse Elemente in $X$, dann besitzt auch $a*b\in X$ ein inverses Element in $X$:\\
		$(a*b)^{-1}=b^{-1}*a^{-1}$
		\item Die Menge der Elemente eines Monoids $(X,*)$, welche ein Inverses besitzen, bilden bezüglich $*$ eine Gruppe.
	\subExEnd

	Bsp. für d): $(\mz,\cdot)$ Elemente mit Inversem in $\mz$\\
	$1,-1$\quad ($\lrc{1,-1,\cdot}$ Gruppe)

	\textbf{Beweis:}
	\subExBegin{a)}
		\item Angenommen, $e_1,e_2$ sind neutrale Elemente in $(X,*)$.$e_1\underset{e_2\mbox{\scriptsize\ neut. El.}}{=}e_1*e_2\underset{e_1\mbox{\scriptsize\ neut. El.}}=e_2$
		\item Angenommen, $a$ besitzt zwei inverse Elemente $b_1, b_2$. Dann $a*b_1=e$ und $b_2*a=e$\\
		$b_1=e*b_1=(b_2*a)*b_1=b_2*(a*b_1)=b_2*e=b_2$
		\item $(a*b)*(b^{-1}*a^{-1})=a*(b*b^{-1})*a^{-1}=(a*e)*a^{-1}=a*a^{-1}=e$\\
		Analog: $(b^{-1}*a^{-1})*(a*b)=e$. Behauptung folgt.
		\item $X*=\lrc{a\in X: a\mbox{ besitzt inverses in }X}$\\
		$(X^*,*)$ Gruppe:
		\subExBegin{1)}
			\item $*$ ist Verknüpfung auf $X^*$. D.h. $a,b\in X^*\Rightarrow a*b\in X^*$, folgt aus c)
			\item Das Assoziativgesetz gilt: in $X*$, da es in $X gilt$.
			\item $e\in X$ liegt auch in $X^*$, da $e*e=e$, d.h. $e^{-1}=e$.
			\item $a\in X^*$, so existiert $a^{-1}\in X$. Tatsächlich: $a^{-1}\in X^*$, dann $\lrr{a^{-1}}^{-1}=a$\\
			$a^{-1}*a=a*a^{-1}=r$
		\subExEnd
	\subExEnd

	\section{Beispiel}

	\subExBegin{a)}
		\item $\mn,\mz,\mq,\mr$ sind Halbgruppe bezüglich der Addition. $\mn_0,\mz,\mq,\mr$ sind Monoide bezüglich Addition (neutrales Element $0$).\\
		$\mz,\mq,\mr$ sind Gruppen bezüglich $+$ (inverses element zu $a$ ist $-a$).
		\item $\mn,\mz,\mq,\mr$ sind Monoide bezüglich Multiplikation (neutrales Element $1$). Keines der $\mn,\mz,\mq,\mr$ ist Gruppe bezüglich der Multiplikation (In $\mz,\mq,\mr$, da $0$ kein Inverses Element besitzt). $\mq\mnot\lrc{0},\mr\mnot\lrc{0}$ Gruppe bezüglich der Multiplikation.
		\item Division: Ist Verknüpfung auf $\mq\mnot\lrc{0}$ oder $\mr\mnot\lrc{0}$. Ist keine Halbgruppe.\\
		$(2:3):4=\frac{2}{3}\cdot\frac{1}{4}\cdot\frac{1}{6}$\\
		$2:(3:4)=2\cdot\frac{4}{3}=\frac{8}{3}$
		\item $X=\mathcal{P}(M)$ mit $\mor$ oder $\mand$ ist Monoid:\\
		Assoziativgesetz: Gilt\\
		$\mvoid$ neutrales Element bezüglich $\mor$. $M$ ist neutrales Element bezüglich $\mand$. Falls $M\neq\mvoid$, ist $\mathcal{P}(M)$ keine Halbgruppe bezüglich $\mnot$.\\
		Wähle $a\in M$.\\
		$(\lrc{a}\mnot\mvoid)\mnot\lrc{a}=\mvoid$\\
		$\lrc{a}\mnot(\mvoid\mnot\lrc{a})=\lrc{a}\neq\mvoid$
		\item Menge aller Abblidungen $M\rightarrow M$ mit Hintereinanderausführungen als Verknüpfung:\\
		Monoid, neutrales Element $id_M$. Genau der bijektiven Abbildungen $M\rightarrow M$ besitzen ein Inverses, die Umkehrabbildung $\alpha^{-1}$. $\alpha\circ\alpha^{-1}=\alpha^{-1}\circ\alpha=id_M$.\\
		Nach 1.5d): Bijektive Abbildung $M\rightarrow M$ bilden bezüglich Hintereinanderausführungen eine Gruppe.
		\item Menge aller endlichen Folgen über Alphabet $A$ mit Konkatenation als Verknüpfung.\\
		Leere Folge (leeres Wort): Neutrales Element. Keine nichtleere Folge hat ein inverses Element.
		\item $n\times n$- Matritzen über $\mr$: $M_n(\mr)$\\
		$(M_n(\mr),+)$ ist abelsche Gruppe. Neutrales Element: Nullmatrix\\
		Inverses zu $A:-A$.\\
		$(M_n(\mr),\underset{\mbox{\scriptsize Matr.mult.}}{\cdot})$ Monoid, neutrales Element\\
		Einheitsmatrix $E_n=\begin{pmatrix}1&\dots&0\\\vdots&1&\vdots\\0&\dots&1
		\end{pmatrix}$\\
		Keine Gruppe. Nullmatrix besitzt kein Inverses Element bezüglich der Multiplikation.\\
		$0\cdot A=0$
		\item $(\mz,\oplus)$\\
		$a\oplus b=(a+b)\mod n$ Gruppe\\
		$(a\oplus b)\oplus c=(((a+b)\mod n)+c)\mod n=((a+b)+c)\mod n$\\
		$a\oplus(b\oplus c)=(a+(b+c)\mod n))\mod n=(a+(b+c))\mod n$\\
		neutrales Element zu $0$: $0$\\
		Inverses zu $a\in\lrc{1,...,n-1}$ ist $-a\mod n=n-a$\\
		$a\oplus(n-a)=(a+n-a)\mod n=n\mod n=0$
		\item $(\mz_n,\odot)$ ist für jedes $n>1$ Monoid.\\
		$a\odot b=a\cdot b\mod n$\\
		neutrales Element: $1$\\
		Keine Gruppe: $0$ ist nicht invers.\\
		$\mz_4=\lrc{0,1,2,3}$\quad $2$ ist nicht invers bezüglich $\odot$.\\
		$2\odot 0=0$\\
		$2\odot 1=2$\\
		$2\odot 2=0$\\
		$2\odot 3=2$\\
		$2$ ist nicht invers bezüglich $\odot$.

		$3\odot 3=3\cdot 3\mod 4=1$\\
		$3$ hat ein inverses Element bezüglich $0$.
	\subExEnd

	\section{Satz: Inverse und \texorpdfstring{$\mz^*_n$}{Z*}}

	$n\in\mn,n>1$.
	\subExBegin{a)}
		\item Die Elemente in $\mz_n$, die bezüglich $\odot$ ein Inverses besitzen, sind genau die $a\in Z_n$ mit $\ggT(a,n)=1$. Für solche $a$ berechnet man das Inverse $a_{-1}$ bezüglich $\odot$ folgendermaßen:\\
		Bestimme mit dem erweiterten Euklidischen Algorithmus $s,t\in mz$ mit $s\cdot a+t\cdot m=1(=\ggT(a,n))$. Dann $a^{-1}=s\mod n$

		\item $Z_n^*:=\lrc{a\in\mz_n:\ggT(a,n)=1}$ ist dann Gruppe bezüglich $\odot$.\\
		$\lrabs{\mz_n^*}=\lrabs{\lrc{0\leq a\leq n:\ggT(a,n)=1}}=\varphi(n)$ \textbf{Euler'sche} $\mathbf{\varphi}$\textbf{- Funktion}\\ nach Leonard Euler, 1707,1783, Basel, St. Petersburg)

		\item Ist $p$ Primzahl, so ist $\mz_p^*=\mz_p\mnot\lrc{0}$ Gruppe bezüglich $\odot$.
	\subExEnd

	\textbf{Beweis:}
	\subExBegin{a)}
		\item Angenommen, $a\in\mz_n$ hat bezüglich $\odot$ Inverses $b\in\mz_n$. Das heißt, $a\odot b=1$. Also $a\cdot b\mod n=1$, das heißt $a\cdot b=k\cdot n+1$ für ein $jk\in\mz$. Sei $d$ ein Teiler von $a$ und von $n$. Da $d|a\Rightarrow d|a\cdot b$, $d|k\cdot n\Rightarrow d|a\cdot b-k\cdot n=1$, d.h. $d=\pm 1$. Also ist $\ggT(a,n)=1$\\
		Umgekehrt: Sei $a\in\mz_n$ und $\ggT(a,n)=1$. Bestimme mit dem erweiterten Euklidischen Algorithmus $s,t\in\mz:s\cdot a+t\cdot n=1\Leftrightarrow 1=(s\cdot a+t\cdot n)\mod n=((s\mod n)\cdot(a\mod n)+(t\mod n)\cdot(n\mod n))\mod n=((s\mod n)\cdot a)\mod n=(s\mod n)\odot a$\\
		Also $a^{-1}=s\mod n$

		\item Folgt aus 1.5 d) und Teil a)
		\item Folgt aus b)
	\subExEnd

	\section{Beispiel}

	$n=8$, $a=5$\\
	$\ggT(8,5)=1$. $a$ hat Inverses in $(\mz_8,\odot)$.
	Erweiterter Euklidischer Algorithmus: $1=(-3)\cdot 5+2\cdot 8$\\
	$5^{-1}=(-3)\mod 8=5$\\
	$5\odot 5=5\cdot 5\mod 8=1$

	\section{Beispiel}

	Sei $M=\lrc{1,...,n}$. Die Menge der bijektiven Abbildungen $N\rightarrow M$ (Permutationen) bilden nach 1.6e) bezüglich der Hintereinanderausführung eine Gruppe. Diese Gruppe wird \textbf{symmetrische Gruppe vom Grad $n$} genannt, $S_n$.\\
	$\lrabs{S_n}=n!=1\cdot 2\cdot 3\cdot...\cdot n$.\\
	$n=3$. $\pi=\begin{pmatrix}1&2&3\\3&2&1\end{pmatrix}\in S_3,\pi^{-1}=\begin{pmatrix}1&2&3\\3&2&1\end{pmatrix}=\pi$

	$\delta=\begin{pmatrix}1&2&3\\2&3&1\end{pmatrix}
	\quad\delta^{-1}=
	\begin{pmatrix}1&2&3\\3&1&2\end{pmatrix}$\\
	$\delta^{-1}\circ\delta=
\begin{pmatrix}1&2&3\\1&2&3\end{pmatrix}=\mbox{id}$\\
	$\pi\circ\delta=\begin{pmatrix}1&2&3\\2&1&3\end{pmatrix},\delta\circ\pi=\begin{pmatrix}1&2&3\\1&3&2\end{pmatrix}$\\
	$\pi\circ\delta\neq\delta\circ\pi$\\
	$S_3$ ist nicht abelsch.

	Ab jetzt: Beliebige Gruppe $(G,\cdot)$\\
	Statt $a\cdot b$ schreibt man $ab$.

	\section{Gleichunglösen in Gruppen}

	Sei $(G,\cdot)$ Gruppe, $a,b\in G$.
	\subExBegin{a)}
		\item Es gibt genau ein $x\in G$ mit $a\cdot x=b$, nämlich $x=a^{-1}\cdot b$ (\gqm{Teilen} von links = Multiplizieren von links mit $a^{-1}$).

		\item Es gibt genau ein $y\in G$ mit $y\cdot a=b$, nämlich $y=b\cdot a^{-1}$ (\gqm{Teilen} von rechts = Multiplizieren von rechs mit $a^{-1}$)

		\item Ist $a\cdot x=b\cdot x$ für ein $x\in G$, so ist $a=b$. Ist $y\cdot a=y\cdot b$ für ein $y\in G$, so ist $a=b$.
	\subExEnd

	\textbf{Beweis:}
	\subExBegin{a)}
		\item $x=a^{-1}\cdot b$ ist Lösung\\
		$a\cdot(a^{-1}\cdot b)=(a\cdot a^{-1})\cdot b=e\cdot b=b$\\
		Eindeutigkeit: $a\cdot x=b\Rightarrow x=e\cdot x=(a^{-1}\cdot a)\cdot x=a^{-1}(a\cdot x)=a^{-1}\cdot b$

		\item Analog zu a)
		\item $a\cdot x=b\cdot x$\\
		Multipliziere beide Seiten von rechts mit $x^{-1}$. $a=(a\cdot x)\cdot x^{-1}=(b\cdot x)\cdot x^{-1}=b$
	\subExEnd

	\section{Definition: Untergruppe}

	Sei $(a,\cdot)$ eine Gruppe, $\mvoid\neq U\mpo G$. $U$ heißt \textbf{Untergruppe} von $G$, falls $u$ bezüglich $\cdot$ selbst eine Gruppe ist.\\
	Bezeichnung: $U\leq G$

	\section{Bemerkung}

	\subExBegin{a)}
		\item Man kann zeigen: Neutrales Element von $U\leq G$ = neutrales Element von $G$. Daher auch: Inverse von Elementen in $U$ = Inverse in $G$.
		\item Um zu zeigen, dass $u\neq\mvoid$ eine Untergruppe von $G$ ist, genügt es, zu zeigen:
		\subExBegin{(1)}
			\item $\forall u,v\in U:u\cdot v\in U$
			\item $\forall u\in U: u^{-1}\in U$
		\subExEnd
		Denn Assoziativgesetz gilt in $U$, da es in $G$ gilt.\\
		Neutrales Element: $\exists u\in U$, da $U\neq\mvoid$. Nach (2): $u^{-1}\in U$. Nach (1): $e=u\cdot u^{-1}\in U$
	\subExEnd

	\section{Satz von Lagrange (Ordnung)}

	Sei $G$ eine endliche Gruppe
	\subExBegin{a)}
		\item \textbf{Satz von Lagrange}\\
		Ist $U\leq G$, so ist $\lrabs{U}\big|\lrabs{G}$\\
		($\lrabs{G}$ = \textbf{Ordnung} von $G$)

		\item Ist $g\in G$, so ist, $g^{\lrabs{G}}=e$\\
		Die kleinste natürliche Zahl $n$ (abhängig von $g$) mit $g^n=e$ heißt \textbf{Ordnung} von $g$.\\
		Man kann $\lrc{g=g^1,g^2,g^3,...,g^n=e}\leq G$ zeigen.\\
		$\lrabs{\lrc{g,g^2,...,g^n}}=n$.

		\textbf{Beweis:} WHK Satz 4.15, 4.22

		\textbf{Beispiel}

		$G=S_3$ (alle Permutationen von $\lrc{1,2,3}$)\\
		Verknüpfung: Hintereinanderausführung.\\
		$g=\lrv{1&2&3\\2&3&1}$ \\
		$g^1 = g \neq\mbox{id}$ \\
		$g^2 = g\circ g =\lrv{1&2&3\\3&1&2}\neq\mbox{id}$\\
		$g^3 = g^2\circ g =\lrv{1&2&3\\1&2&3}=\mbox{id}$\\
		Ordnung von $g: 3$
	\subExEnd
\section{Sätze von Euler und Fermat}
	\subExBegin{a)}
		\item Satz von Euler\\
			Sei $n\in\mn, a\in\mz$\\
			Es gelte $\ggT\lrr{a,n}=1$\\
			Dann gilt: $a^{\phi(n)}\equiv 1\lrr{\mod n}$\\
			$\phi(n)=\lrabs{\lrc{b\in\mn: b\leq n, \ggT\lrr{b,n}=1}}$
		\item Kleiner Satz von Fermat (Fermat 1601-1665)\\
			Ist $p$ eine Primzahl, so gilt\\
			$a^{p-1}\equiv 1\lrr{\mod p}$ für alle $a\in\mz$ mit $p\nmid$
	\subExEnd
	\textbf{Beweis}
	\subExBegin{a)}
		\item Zeige\\
			$a^{\phi(n)}\mod n = 1$\\
			$a^{\phi(n)}\mod n = \underbrace{\lrr{a\mod n}}^{\phi(n)}\mod n$\\
			O.B.d.A ist $1\leq a < n$\\
			Nach 1.7 ist $a\in\lrr{\mz_n^*,\odot}$, da $\ggT\lrr{a,n}=1$\\
			1.13.b) $a^{\lrabs{\mz^*}}=1$ (Potenz bezüglich $\oplus$)\\
			$a^{\lrabs{\mz_n^*}}\mod n = 1$ (normale Multiblikation)\\
			$\lrabs{\mz_n^*}=\phi\lrr{n}$
		\item folgt aus a) mit $\phi(p) = p-1$
	\subExEnd
\section{Anwendung: Das RSA-Verschlüsselungsverfahren}
	\textit{Siehe Skript Codierung} \\
	Kryptologie Verschlüsselung von Daten zur Geheimhaltung.\\
	$\underset{m}{\mbox{\fbox{A}}}\overset{c}{\longrightarrow}$\fbox{B}
	\subExBegin{A:}
		\item $E\lrr{m,k_v}=c$ ($k_v$ ist der Verschlüsselungs-Schlüssel)
		\item $D\lrr{c,k_e}=m$ ($k_e$ ist Entschlüsselungs-Schlüssel)
	\subExEnd
	Ist $k_v = k_e$ spricht man von symmetrischen Verschlüsselungsverfahren. Hier muss der Schlüssel geheim gehalten werden.

	1976 entwarfen Diffie und Hellman das Asymmetrische Verschlüsselungsverfahren oder auch Public-Key-Verfahren.\\
	\fbox{A}$\overset{m}{\longrightarrow}$\fbox{B} \\
	$k_v$ ist öffentlich, $k_e$ geheim.
	\subExBegin{A:}
		\item $E\lrr{m,k_v}=c$
		\item $D\lrr{c,k_e}=m$
	\subExEnd
	Eine Einweg-Funktion ist schnell zu berechnen. Die Umkehrfunktion soll nur mit sehr großem Aufwand berechenbar sein.\\
	Aber mit der Zusatzinformation $k_e$ ist die Codierung leicht zu invertieren!\\
	Insbesondere muss $k_e$ sehr schwer aus $k_v$ zu bestimmen sein.

	Erstmals konkret realisiert wurde das RSA-Verfahren 1977 von Revest, Shamir und Adlerman.\\
	\textbf{B: Schlüsselerzeugung}\\
	$B$ wählt zwei große Primzahlen $p,q,p\neq q$ (1000 Bit)
	(Miller-Rabin-Test)\\
	$n=p\cdot q$\\
	$\phi(n)=\lrr{p-1}\cdot\lrr{q-1}$\\
	Erklärung:\\
	Zahlen $\leq n$, die nicht teilerfremd zu $n$ sind:\\
	$p,2p,3p,4p,\dots,\lrr{q-1}\cdot p$\\
	$q,2q,3q,4q,\dots,\lrr{p-1}\cdot q$\\
	$q\cdot p = n$\\
	Die Anzahl beträgt also $q-1+p-1+1 = p+q-1$\\
	teilerfremd:
	$n-p-q+1=p\cdot q-p-q+1=\lrr{p-1}\cdot\lrr{q-1}$

	$B$ wählt $1\leq e<\phi(n)$ mit $\ggT\lrr{e,\phi\lrr{n}}=1$\\
	(Zufallswahlen und Euklidischer Algrorythmus)\\
	\textbf{Öffentlicher Schlüssel} von $B:\lrr{n,e}$\\
	$B$ berechnet $d$ mit $e\cdot d\equiv 1\lrr{\mod\phi\lrr{n}}$\\
	$e\cdot d\mod\phi\lrr{n}=1$\\
	$ed=1+k\cdot\phi\lrr{n}$\\
	geheimer Schlüssel von $B:d$

	\textbf{Verschlüsselung}\\
	$A\overset{m}{\rightarrow}B$\\
	$m$ ist binär codiert.\\
	Zerlege $m$ in Blöcke, jeder Block wird als Binärzahl gelesen $<n$\\
	Diese Blöcke werden einzeln verschlüsselt\\
	$m<n$\\
	$m^e\mod n=c$

	\textbf{Entschlüsselung}\\
	$c^d\mod n = m$\\
	Warum gilt das?\\
	Angenommen der $\ggT\lrr{m,n}=1$\\
	$m^{\phi\lrr{n}}\mod n=1$ (Satz von Euler)\\
	$\lrr{m^{\phi\lrr{n}}}^k\mod n = 1$\\
	$m^{k\cdot\phi\lrr{n}+1}\mod n = m$\\
	$m^{ed}\mod n = c^d\mod n=m$

	\textbf{Sicherheit}
	\begin{itemize}
		\item $e$-te Wurzel $\mod n$ ist schwierig
		\item Kann ein Angreifer aus $\lrr{n,e}\;\;d$ bestimmen?\\
			Das geht genau dann, wenn man $\phi(n)$ kennt. (EEA)
		\item $\phi(n)$ per Definition durch Abzählen\\
			$n$ ist eine $2000$-Bit-Zahl.\\
			Angenommen $10^9$ Zahlen in der Sekunde werden getestet, ob sie teilerfremd zu $n$ sind (EA).\\
			Dann benötigt man $\dfrac{2^1000}{10^9}$sec$\approx 10^{585}$ Jahre
		\item $\phi(n)$ ist leicht zu bestimmen, falls $p,q$ bekannt sind.\\
			Das heist ein Angreifer muss $n$ faktorisieren\\
			Es ist kein schnelleres Verfahren bekannt.
	\end{itemize}
\section{Definition: Ring}
	$R\neq\mvoid$ mit zwei Verknüpfungen $+$ und $\cdot$
	\subExBegin{a)}
		\item $\lrr{R,+,\cdot}$ heißt \textbf{Ring}, falls gilt:
			\subExBegin{(1)}
				\item $\lrr{R,+}$ ist eine kommutative Gruppe\\
					neutrales Element: 0, \textbf{Null, Nullelement}\\
					inverses Element zu $a\in R$ bezüglich $+$: $-a$\\
					$a+\lrr{-b} =: a-b$
				\item $\lrr{R,\cdot}$ ist eine Halbgruppe
				\item $\forall a,b,c\in R$:\\
					$a\cdot\lrr{b+c}=a\cdot b+a\cdot c$\\
					$\lrr{a+b}\cdot c=a\cdot c+b\cdot c$\\
					(Es gilt Punkt vor Strich)\\
					\textbf{Distributivgesetze}
			\subExEnd
		\item $\lrr{R,+,\cdot}$ heißt \textbf{kommutativer Ring}, falls $\forall a,b\in R:a\cdot b=b\cdot a$
		\item $\lrr{R,+,\cdot}$ heißt \textbf{Ring mit Eins}, falls $\lrr{R,\cdot}$ ein Monoid ist mit neutralem Element $1\neq 0$.\\
			\textbf{Eins, Einselement}
	\subExEnd
\section{Beispiele}
	\subExBegin{a)}
		\item $\lrr{\mz,+,\cdot}$ ist ein kommutativer Ring mit Eins.\\
			Nur $1$ und $-1$ haben Inverse bezüglich $\cdot$
		\item $\lrr{\mq,+,\cdot}$,$\lrr{\mr,+,\cdot}$ sind kommutative Ringe mit Eins.\\
			Jedes Element $\neq 0$ hat ein Inverses bezüglich $\cdot$
		\item $\lrr{\mz_n,\oplus,\odot}\quad n>1$\\
			$\mz_n=\lrc{0,1,\dots,n-1}$\\
			$\lrr{\mz_n,\oplus}$ ist kommutative Gruppe\\
			$\lrr{\mz_n,\odot}$ ist Monoid\\
			Distributivgesetze\\
			$a\odot\lrr{b\oplus c}=a\odot\lrr{\lrr{b+c}\mod n}\\
			=\lrr{a\cdot\lrr{\lrr{b+c}\mod n}}\mod n=\lrr{a\cdot\lrr{b+c}}\mod n\\
			=\lrr{a\cdot b+a\cdot c}\mod n=\lrr{\lrr{a\cdot b}\mod n+\lrr{a\cdot c}\mod n}\mod n\\
			=\lrr{a\odot b+a\odot c}\mod n=a\odot b\oplus a\odot c$
		\item $n>1\quad R=$ Menge der $n\times n$-Matrizen über $\mr$ mit Matrizenaddition $+$ und Matrizenmultiplikation $\cdot$\\
			$\lra{n=2:\lrv{1&2\\4&5}+\lrv{-1&5\\1/2&7}=\lrv{0&7\\9/2&12}\\
				\lrv{1&2\\4&5}+\lrv{-1&5\\1/2&7}=\lrv{0&19\\-3/2&55}}$ \\
			Ring mit Eins:\\
			Nullelement: $0=\lrv{0&\dots&0\\\vdots&&\vdots\\0&\dots&0}$\\
			Einselement: Einheitsmatrix $E_n=\begin{pmatrix}1&\dots&0\\\vdots&1&\vdots\\0&\dots&1
		\end{pmatrix}$\\
			$R$ ist \textbf{kein} kommutativer Ring.
	\subExEnd
\section{Proposition}
	Sei $\lrr{R,+,\cdot}$ ein Ring, Dann gilt für alle $a,b\in R$
	\subExBegin{a)}
		\item $0\cdot a=a\cdot 0=0$
		\item $\lrr{-a}\cdot b=a\cdot\lrr{-b}=-\lrr{a\cdot b}$
		\item $\lrr{-a}\cdot\lrr{-b}=a\cdot b$
	\subExEnd
	\textbf{Beweis}
	\subExBegin{a)}
		\item $0\cdot a =\lrr{0+0}\cdot a$ \\
			Addittion auf beiden Seiten $-(0\cdot a)$\\
			$0=0\cdot a+0=0\cdot a$\\
			$a\cdot 0=0$ analog
		\item Zeige: $\lrr{-a}\cdot b=-\lrr{a\cdot b}$\\
			$\lrr{-a}\cdot b+a\cdot b=\lrr{-a+a}\cdot b = 0\cdot b \underset{\mbox{\scriptsize a)}}{=} 0$\\
			$\lrr{-a}b$ ist das Inverse bezüglich $+$ von $a\cdot b$\\
			$\lrr{-a}\cdot b=-\lrr{a\cdot b}$\\
			Analog $a\cdot\lrr{-b}=-\lrr{a\cdot b}$
		\item $\lrr{-a}\cdot\lrr{-b}\underset{\mbox{\scriptsize b)}}{=}-\lrr{a\cdot\lrr{-b}}\underset{\mbox{\scriptsize b)}}{=}-\lrr{-\lrr{a\cdot b}}=a\cdot b$
	\subExEnd
\section{Bemerkung}
	\subExBegin{a)}
		\item In jedem Ring mit Eins sind $1$ und $-1$ invertierbar bezüglich $\cdot$\\
			$1\cdot 1=1\quad 1^{-1}=1$\\
			$\lrr{-1}\cdot\lrr{-1}=1\quad\lrr{-1}^{-1}=-1$\\
			Es kann noch mehr bezüglich der Multiplikation inverse Elemente geben.\\
			Es kann auch $1=-1$ sein.\\
			z.B. In $\lrr{\mz_2,\oplus,\odot}: -1=1\quad 1\oplus 1=0$
		\item In einem Ring mit $1$ ist $0$ \textbf{nie} invertierbar bezüglich $\cdot$, denn $0\cdot a=0\neq 1$ für alle $a\in R$
	\subExEnd
\section{Definition: Körper}
	Ein kommutativer Ring $\lrr{R,+,\cdot}$ mit Eins heißt \textbf{Körper} (englisch field), wenn jedes $a\neq 0$ in $R$ ein Inverses $a^{-1}\in R$ bezüglich $\cdot$ besitzt.
\section{Beispiele}
	\subExBegin{a)}
		\item $\mq,\mr,\mc$ Körper\\
			$\mz$ nicht.
		\item $n>1\quad\mz_n=\lrc{0,\dots,n-1},\oplus,\odot$ ist genau dann ein Körper, wenn $n$ eine Primzahl ist. (1.7)\\
			Besonders wichtig ist $\mz_2=\lrc{0,1}$
	\subExEnd
\section{Proposition: Nullteilerfreiheit in Körpern}
	Ist $K$ ein Körper und sind $a,b\in K$ mit $a\cdot b=0$, so ist $a=0$ oder $b=0$

	\textbf{Beweis}

	Angenommen $a\neq 0$. Zeige $b=0$\\
	$b=1\cdot b \underset{a\neq 0}{=} \lrr{a^{-1}\cdot a}\cdot b=a^{-1}\lrr{a\cdot b}=a^{-1}\cdot 0\underset{\mbox{\scriptsize 1.18a)}}{=}0$

	\textbf{Beispiel}

	$\lrr{\mz_6,\oplus,\odot}$ ist kein Körper $\lrr{2,3\neq 0 \;\mbox{aber}\; 2\odot 3=6\mod 6 =0}$
\section{Definition von Polynomringen}
	Sei $K$ ein Körper.
	\subExBegin{a)}
		\item Ein \textbf{Polynom} über $K$ ist ein Ausdruck.\\
			$f=a_0+a_1x+\dots+a_nx^n\quad n\in\mn_0\quad a_i\in K$\\
			(Manchmal $f(x)$ statt $f$)\\
			Ist $a_i=0$, so kann man $0\cdot x^i$ weglassen.\\
			Statt $a_0x^0$ schreibt man $a_0$\\
			Sind alle $a_i=0$ heißt $f$ \textbf{Nullpolynom} $f=0$\\
			Ist $a_i = 1$, so schreibt man auch $x^i$ statt $1\cdot x^i$
		\item Zwei Polynome sind \textbf{gleich}, falls entweder $f=g=0$ oder $f,g\neq 0$.\\
      ($f=a_0+a_1x+\dots+a_nx^n,a_n\neq 0\\
      g=b_0+b_1x+\dots+b_mx^m,b_m\neq 0$)\\
			$n=m, a_i=b_i, i=0,\dots,n$
		\item Konventionelle Reihenfolge der $a_ix^i$ kann verändert werden, ohne dass sich $f$ ändert.\\
			$x^7+x^2+1=x^2+1+x^7$
		\item $K\lra{x}=$ Menge aller Polynome über $K$
	\subExEnd
	\textbf{Beispiel}

	$K$ Körper, $f=a_0a_1x+\dots+a_nx^n\quad a_i\in k\quad n\in\mn_0$ formaler Ausdruck, Polynom\\
	$f=3x^2+5x+2$\\
	$g=-x^3-x+1$\\
	$f,g\in\mq\lra{x}$\\
	$f+g=-x^3+3x^2+4x+3$\\
	$f\cdot g=\lrr{3x^2+4x+2}\lrr{-x^3-x+1}=-3x^5-5x^4-5x^3-2x^2+3x+2$
\section{Satz und Definition: Polynomring}
	Sei $K$ ein Körper.\\
	$K\lra{x}$ wird zu einem kommutativen Ring mit Eins durch folgende Verknüpfungen:\\
	$f=\limsum{i=0}{n}a_ix^i, g=\limsum{j=0}{m}b_jx^j$\\
	$f+g = \limsum{i=0}{\mbox{\scriptsize max}(n,m)}\lrr{a_i+b_i}x^i$\\
	$f\cdot g= \limsum{i=0}{n+m} c_ix^i$, wobei $c_i=\limsum{j=0}{i}a_jb_{i-j}$ für $i=0,\dots,n+m$\\
	(Faltungsprodukt)\\
	In beiden Fällen sind die Koeffizienten $a_i$ mit $i>n$ oder $b_j$ mit $j>m$ gleich $0$ zu setzen.\\
	Das Einselement ist das Polynom 1\\
	$\lrr{a_0=1,a_i=0\mbox{ für }i\geq 1}$\\
	Das Nulleelement ist das Nullpolynom\\
	$\lrr{K\lra{x},+,\cdot}$\\
	\textbf{Polynomring} in einer Variablen.

	\textbf{Beweis}

	$c_{n+m}=\limsum{j=0}{n+m}a_jb_{n+m-j}=\limsum{j=0}{n}a_jb_{n+m-j}=a_nb_m$
\section{Bemerkung}
	\subExBegin{a)}
		\item $f=\limsum{i=0}{n}a_ix^i\quad a\in K$\\
			$\underset{1,24}{a\cdot f} =\limsum{i=0}{n}aa_ix^i$\\
			$\underset{1.24}{x\cdot f} =\limsum{i=0}{n}a_ix^{n+1}$
		\item Das $+$-Zeichen bei der Beschreibung von Polynomen entspricht der Addition der \textit{Monome} $a_ix^i$
	\subExEnd
\section{Definition: Grad}
	Sei $0\neq f\in K\lra{x}$\\
	$f=\limsum{i=0}{n}a_ix^i$ mit $a_n\neq 0$\\
	$n=\grad(f)$ \textbf{Grad} von $f$.\\
	$\grad\lrr{0}:=-\infty$
\section{Satz: Gradformel}
	Sei $K$ ein Körper, $f,g\in K\lra{x}$.\\
	Dann ist $\grad\lrr{f\cdot g}=\grad\lrr{f}+\grad\lrr{g}$\\
	Konvention:\\
	$-\infty+-\infty=-\infty$\\
	$-\infty+n=-\infty$

	\textbf{Beweis}

	Formel ist richtig, falls $f=0$ oder falls $g=0$.\\
	$f:\limsum{i=0}{n}a_ix^i\quad a_n\neq 0$\\
	$g:\limsum{j=0}{m}b_jx^j\quad b_m\neq 0$\\
	$\lrr{f\cdot g = 0}$\\
	Sei $f\neq 0$ und $g\neq 0$\\
	Der höchste Koeffizient in $f\cdot g$ ist $a_nb_m$\\
	$f\cdot g=a_nb_mx^{n+m}+\dots$\\
	$a_n,b_m\neq 0\underset{1.22}{\Rightarrow}a_n\cdot b_m\neq 0$\\
	$\grad\lrr{f\cdot g}=n+m=\grad\lrr{f}+\grad\lrr{g}$
\section{Korollar}
	Genau die Polynome von Grad $0$ (konstante Polynome $\neq 0$) haben Inverse bezüglich $\cdot$ in $K\lra{x}$

	\textbf{Beweis}

	Sei $f\neq 0$. Angenommen $f$ besitzt ein Inverses $f^{-1}$ in $K\lra{x}$\\
	$f\cdot f^{-1} =1$\\
	$\grad(f)+grad\lrr{f^{-1}}=\grad\lrr{f\cdot f^{-1}}=\grad(1)=0\Rightarrow\grad(f)=\grad\lrr{f^{-1}}= 0$
\section{Bemerkung}
	$f=a_0+a_1+x+\dots+a_nx^n$
	\subExBegin{a)}
		\item Jedem $f\in K\lra{x}$ kann man Funktion $K\rightarrow K$ zuordnen:\\
			$a\in K\mapsto f(a):=a_0+a_ia+\dots+a_na^n\quad \in K$\\
			Aufgrund der Definition von Addition und Multiplikation in $K\lra{x}$ gilt:\\
			$\underbrace{\lrr{f+g}}_{K\lra{x}}(a)=f(a)\underbrace{+}_{K}g(a)$\\
			$\underbrace{\lrr{f\cdot g}}_{K\lra{x}}\lrr{a}=f(a)\underbrace{\cdot}_K g(a)$

			Es kann passieren, dass zwei verschiedenen Polynomen die gleiche Funktion zugeordnet wird:\\
			$f=x\quad g=x^2$ über $K=\mz_2$\\
			$f(0)=0\quad f(1)=1$\\
			$g(0)=0\quad g(1)=1$\\
			Bei unendlich $K$ passiert das nicht (später!)
		\item Wie berechnet man $f(a)=a_0+a_1a+a_2a^2+\dots +a_na^n$ am schnellsten?\\
			\textbf{Homer-Schema}\\
			$f(a) = a_0+a(a_1+a(a_2+\dots a(a_{n-1}+aa_n)\dots))$
	\subExEnd
\section{Definition: Division von Polynomen}
	$K$ Körper, $0\neq f\in K\lra{x}, g\in K\lra{x}$\\
	$f$ \textbf{teilt} $g$, $f|g$, falls Polynom $q\in K\lra{x}$ existiert mit $g=q\cdot f$.\\
	(Wenn $g\neq 0\quad \grad(g)\geq\grad(f)$)
\section{Satz: Division mit Rest}
	$K$ Körper, $f,g\in K\lra{x}, f\neq 0$\\
	Dann existieren eindeutig bestimmte Polynome $q,r\in K\lra{x}$ mit\\
	$g=q\cdot f+r$ und $\grad(r) <\grad(f)$\\
	Bezeichnung $r=g\mod f, q=g\div f$
	\begin{flushright}
		(WHK, Satz 4.69)
	\end{flushright}
\section{Beispiele}
	\subExBegin{a)}
		\item $g=x^4+2x^3-x+2\in\mq\lra{x}$\\
			$f=3x^2-1\in\mq\lra{x}$

			$\begin{array}{rrrrrl}
				\underline{x^4}	&+2x^3				&						&-x				&+2			&: \underline{3x^2}-1=\overbrace{\frac{1}{3}x^2+\frac{2}{3}x+\frac{1}{9}}^{q}\\
				-(x^4			&					&-\frac{1}{3}x^2)		&				&&\\
								&\underline{2x^3}	&+\frac{1}{3}x^2			&-x			&+2&\\
								&-(2x^3				&						&-\frac{2}{3}x)	&&\\
								&					&\frac{1}{3}x^2			&-\frac{1}{3}x	&+2	&\\
								&					&-(\frac{1}{3}x^2		&				&-\frac{1}{9}	&\\
								&					&						&\underbrace{-\frac{1}{3}x+\frac{19}{9}}_r&
			\end{array}$	
		\item $g=x^4-x^2+1\quad f=x^2+x$\\
			$\lrr{g=x^4+2x^2+1}\quad f,g\in \mz_3\lra{x}$

			$\begin{array}{rrlrl}
				x^4		&		&+2x^2	&+1		&:x^2+x=\overbrace{x^2+2x}^{q}\\
				-(x^4	&+x^3)			&\\
						&2x^3	&+2x^2	&+1\\
						&-(2x^3	&+2x^2)	&\\
						&		&		&\underbrace{1}_{r}
			\end{array}$

			$x^4+2x^2+1 =\lrr{x^2+2x}\lrr{x^2+x}+1$ in $\mz_2\lra{x}$
	\subExEnd

	\section{Korollar: Teilbarkeit durch x-a}

	$K$ Körper, $a\in K$, $f\in K[x]$\\
	$f$ ist genu dann durch $x-a$ teilbar, wenn $f(a)=0$ (d.h. $a$ ist Nullstelle von $f$)\\
	\textbf{Beweis:} Angenommen $(x-a)|f$, d.h. $f=q(x-a),q\in K[x]$. $f(a)=q(a)(a-a)=q(a)\cdot 0=0$.\\
	Angenommen $f(a)=0$. Division mit Rest von $f$ durch $(x-a)$.\\
	$f=q\cdot(x-a)+r$. $\grad r<\grad(x-a)=1$. $r$ ist konstantes Polynom. Einsetzen von $a$.\\
	$0=f(a)=q(a)(a-a)+r=0+r=r$\\
	Also: $f=q\cdot(x-a),\ x-a|f$

	\section{Definition: m-fache Nullstelle}

	$a\neq f\in K[x],a\in K$ heißt \textbf{$m$- fache Nullstelle} (oder \textbf{Nullstellel mit Vielfachheit $m$}) von $f$, falls $(x-a)^m|f$ und $(x-a)^{m+1}\nmid f$, $q(a)\neq 0$

	$f=q\cdot(x-a)^m$, $q(a)\neq 0$

	\section{Lemma}

	Sei $K$ ein Körper
	\subExBegin{a)}
	\item $K[x]$ ist ein nullteilerfreier Ring, d.h. sind $f,g\in K[x]$ mit $f\cdot g=0$, so ist $f=0$ oder $g=0$.
	\item Sind $f,g_1,g_2\in K[x]$ mit $f\cdot g_1=f\cdot g_2$, $f\neq 0$, so ist $g_1=g_2$
	\subExEnd

	\textbf{Beweis}
	\subExBegin{a)}
	\item Gradformel 1.27: $f\cdot g=0$\\
	$-\infty=\grad(0)=\grad(f\cdot g)=\grad(f)+\grad(g)$\\
$	\grad(f)=-\infty$ oder $\grad(g)=-\infty$
\item $f\cdot g_1=f\cdot g_2$ $\Rightarrow$\\
$f(g_1-g_2)=0\underset{\scriptstyle{a),f\neq 0}}{\Rightarrow} g_1-g_2=0$, d.h. $g_1=g_2$
	\subExEnd

	\section{Satz zu Nullstellen}

	Sei $K$ ein Körper, $f\in K[x]$.\\
	$\grad(f)=m\geq 0$\\
	Dann hat $f$ höchstens $m$ Nullstellen (einschließlich Vielfachheiten) in $K$.\\
	Genauer: Sind $a_1,...,a_k$ die verschiedenen Nullstellen von $f$ mit Vielfachheiten $m_1,...,m_k$, dann ist $f=q\cdot(x-a_1)^{m_1},...,(x-a)^{m_k}$.\\
	$q\in K[x]$ ohne Nullstellen in $K$. $m_1+...+m_k\leq m$.

	\textbf{Beweis:} Induktion nach $m$\\
	$m=0$: $f$ konstantes Polynom $\neq 0$, keine Nullstelle.\\
	$M>0$: Induktionvorraussetzung: Aussage ist richtig für alle Polynome vom Grad $<m$.\\
	$a_1$ ist Nullstelle von Vielfachheit $m_1$ von $f$.\\
	$f=q(x-a_1)^{m_1},q(a_1)\neq 0$.\\
	$\grad(q)\underset{\mbox{\scriptsize 1.72}}{=}m-m_1<m$\\
	Wir zeigen: $q$ hat genau die Nullstellen $a_2,...,a_k$ mit Vielfachheit $m_2,...,m_k$.

	Klar. Jede Nullstelle von $q$ ist Nullstelle von $f$.\\
	$i\in\lrc{2,...,k}$\\
	$0=f(a_i)=q(a_i)\underbrace{(a_i-a_1)^{m_1}}{\neq 0}\underset{\mbox{\scriptsize 1.22}}{\Rightarrow} q(a_i)=0$.\\
	Die Nullstellen von $q$ sind genau die $a_2,...,a_k$.

	Die Vielfachheit von $a_i,i\in\lrc{2,...,k}$, als Nullstelle von $q$ ist höchstehs $m_i$.\\
	($q=(x-a_i)^n\cdot\tilde{q}\Rightarrow f=(x-a_i)\cdot\tilde{q}\cdot(x-a_i)^m$)

	Zeige: Für $i\in\lrc{2,...,k}$ besitzt $q$ die Nullstelle $a_i$, genau mit Vielfache $m_i$

	$f(x)=q(x)(x-a_i)^{m_1}$\\
	$f(x)=s(x)(x-a_i)^{m_i}$\\
	Da $a_i$ Nulstelle von $q$,\\
	$q=q_1\cdot(x-a_i)$

	$f(x)=q_1(x)\underline{(x-a_1)}(x-a_1)^{m_1}=s(x)\underline{(x-a_i)}^{m_i}$\\
	1.35b): Kürze $(x-a_i)$\\
	$q_1(x)=(x-a_1)^{m_1}=s(x)\cdot(x-a_i)^{m_i-1}$\\
	Falls $m_i-1>0$, so $q_1(a_i)(a_i-a_1)^{m_1}=0$

	Also: $q_1(a_i)=0$\\
	$q_1(x)=q_2\cdot(x-a_i)$\\
	$q(x)=q_2(x)(x-a_1)^2$

	$q_2(x)(x-a_i)(x-a_i)^{m_1}=s(x)(x-a_i)^{m_i-1}$\\
	$q_2(x)(x-a_i)^{m_1}=s(x)(x-a_i)^{m_i-2}$\\
	So fortfahrend:\\
	$q(x)=q_{m_i}(x)(x-a_i)^{m_i}$

	Also: $g(x)$ hat Nullstelle $a_i$ mit Vielfachheit $m_i$, $i=2,...,k$\\
	Da $\grad(q)<\grad(f)=m$ folgt per Induktion $q(x)=g(x)(x-a_2)\dots(x-a_k)^{m_k}$ $g(x)$ ohne Nullstellen in $K$.\\
	$f(x)=g(x)(x-a_1)^{m_1}(x-a_2)^{m_2}\dots(x-a_k)^{m_k}$

	\section{Korollar}

	$K$ Körper, $f,g\in K[x]$, $m\max(\grad(f),\grad(g))$\\
	Gibt es $m+1$ viele paarweise verschiedene Elemente $a_1,...,a_{m+1}\in K$ mit $f(a_i)=g(a_i)$, $i=1,..,m+1$, dann ist $f=g$. Insbesondere ist $\lrabs{K}=\infty$, $f,g\in K[x]$ mit $f(a)=g(a)$ für alle $a\in K$, so ist $f=g$.

	\textbf{Beweis:}\\
	$\grad(f-g)\leq m$.\\
	$(f-g)(a_i)=f(a_i)-g(a_i)=0$\\
	$f-g$ hat mindestens $m+1$ viele Nullstellen\\
	1.36: $f-g=0,f=g$\\
	$k=\mq:$ $3x-7=5\lrr{\frac{3}{5x-\frac{7}{5}}}$

	\section{Definition: irreduzibel}

	Ein Polynom $p\in K[x],\grad(p)\geq 1$ heißt \textbf{irreduzibel}, falls folgendes gilt:\\
	Ist $p=f\cdot g$, $f,g\in K[x]$, so ist $\grad(f)=0$ oder $\grad(g)=0$.\\
	(Zerlegung $p=f\cdot g$ mit $\grad(f)=0$ ist immer möglich: $f=a\in K,a\neq 0$. $p=a\cdot(a^{-1}\cdot p)$.

	\section{Beispiel}

	\subExBegin{a)}
	\item $ax+b$, $a\neq 0$\\
	$a,b\in K$ ist irreduzibel (für jeden Körper $K$), denn $ax+b=f\cdot g$. Gradformel: $1=\underbrace{\grad(f)}_{\scriptstyle \geq 0}+\underbrace{\grad(g)}{\scriptstyle \geq 0}$
	\item $x^2-2\in\mq[x]$ ist irreduzibel in $\mq[x]$:\\
	Angenommen, dies wäre nicht der Fall. Dann folgt aus der Gradformel: $x^2-2=(ax+b)(cx+d)$ mit $a,b,c,d\in\mq,a,c\neq 0$\\
	$ax+b$ hat Nullstelle: $-a^{-1}\cdot b\in\mq=-\frac{b}{a}$.\\
	$\Rightarrow x^2-2$ hat Nullstelle in $\mq$. $x^2-2$ hat Nullstellen $\pm\sqrt{2}$ in $\mr$.\\
	Aber: $\sqrt{2},-\sqrt{-2}\notin\\mq$\\
	$x^2-2$ ist in $\mr[x]$ nicht irreduzibel, denn $x^2=(x-\sqrt{2})(x+\sqrt{2})$
	\item $x^2+1\in\mr[x]$ ist irreduzibel:\\
	Angenommen nicht: $x^2+1=f\cdot g,\ \grad(f)=\grad(g)=1$,
	d.h. $x^2+1$ hat Nullstelle in $\mr$. Das geht nicht in $\mr$.
	\item $x^2+1\in\mz_5[x]$ ist nicht irreduzibel: $x^2+1$ hat Nullstellen, nämlich $2,3\in\mz_5$ $(x^2+1)=(x-2)(x-3)$
	\item $x^3+x+1\in\mz_2[x]$\\
	Angenommen, nicht irreduzibel:\\
	$x^3+x+1=f\cdot g,\grad(f)=\grad(g)>0$\\
	$3=\grad(f)+\grad(g)$\\
	o.B.d.A. $\grad(f)=1$, d.h. $f$ hat Nullstelle, also auch $x^3+x+1$ (in $\mz_2$). Aber $x^3+x+1$ hat keine Nullstellen in $\mz_2$. $x^3+x+1$ ist irreduzibel in $\mz_2[x]$.
	\subExEnd

	\section{Bemerkung: normiert}

	$f\neq 0$, $f\in K[x]$ hat \textbf{normiert}, falls der höchste von Null verschiedene Koeffizient $1$ ist.\\
	($x^7-3x^3+1$ normiert $-2x^3+2x-1=-2(x^3-x+\frac{1}{2}$ nicht normiert.\\
	Es gilt: Ist $f\neq 0$, $f\in K[x]$, so existieren eindeutig bestimmte norminierte irrduzible Polynome $P_1,...,P_r$ (nicht notwendig verschieden) und $a\in K$ mit $f=a\cdot P_1\cdot P_2\cdot\dots\cdot P_r$\\
	(In $\mz$: $(\pm 1)\cdot \underbrace{P_1\cdot\dots\cdot P_r}_{\mbox{\scriptsize Primzahlen}}$)
