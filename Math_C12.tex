\newpage
\section{Taylorpolynome und Taylorreihen}
	$f$ diffbar an Stelle $c$:\\
	$f$ wird an Stelle $c$ durch Tangente (Polynom vom $\grad\leq 1$) gut approximiert:\\
	$t_1\lrr{x}=f(x)+f'(c)(x-c)$ Tangentengleichung\\
	$\limlim{x\rightarrow c}\dfrac{f(x)-t_1(x)}{x-c}=0$\\
	$t_1\lrr{c}=f(c),\;\; t'(c)=f'(c)$
	
	\textbf{Ziel}\\
	Verbesserung der Approximation von $f$ druch Polynome $t_n$ vom $\grad\leq n$, sodass alle Ableitungen von $f$ und $t_n$ (bis zur $n$-ten Ableitung) an der Stelle $c$ übereinstimmen.
	
	\textbf{Bezeichnung}\\
	$f^{\lrr{n}}$ $n$-te Ableitung von $f$ (fallls $f$ $n$-mal diffbar)\\
	$f'=f^{(1)}, f''=f^(2)$
	
\subsection{Satz und Definition}
	$I$ Intervall, $f:I\rightarrow\mr$ $n$-mal diffbar, $c\in I$\\
	Dann gibt es Polynom $t_n$ vom $\grad\leq n$ (das von $f$ und $c$ abhängt) mit $t^{(i)}\lrr{c}=f^{(i)}\lrr{c}$, $i=0,\dots, n$\\
	Es ist $t_n\lrr{x}=f(c)+f'(c)(x-c)+\frac{f''(c)}{2}\lrr{x-c}^2+\frac{f'''(c)}{3!}\lrr{x-c}^3+\dots+\frac{f^{(n)}(c)}{n!}\lrr{x-c}^n$\\
	$t_n$ heißt $n$-tes Taylorpolynom von $f$ an Entwicklungspunkte $c$
	
	\textbf{Beweis}
	
	Ansatz für $t_n:\limsum{i=0}{n}a_i\lrr{x-c}^i$\\
	(jedes Polynom vom $\grad\leq n$ lässt sich so schreiben)\\
	$t_n^{(j)}\lrr{c}=a_j\cdot j!\ouset{=}{!}{} f^{(j)}\lrr{c}$\\
	Wähle $a_j=\frac{f^{(j)}(c)}{j!}, j=0,\dots, n$
	
\subsection{Satz}
	$I$ Intervall, $f:I\rightarrow\mr$ $n$-mal stetig diffbar (das heißt $f$ $n$-mal diffbar und $f^{(n)}$ ist stetig), $c\in I$, $t_n$ $n$-tes Taylorpolynom zu $f$ an $c$.\\
	Ist $f(x)=t_n(x)+R(x)$, so gilt das \gqm{Restglied} $R(x)$\\
	$\limlim{x\rightarrow c}\frac{R(x)}{(x-c)^n}=0$, das heißt $\limlim{x\rightarrow c}\frac{f(x)-t_n(x)}{(x-c)^n}=0$
	
	\textbf{Beweis}
	
	$\limlim{x\rightarrow c}\dfrac{f(x)-t_n(x)}{(x-c)^n}\ouset{=}{\smt{L'Hôpital}}{}\limlim{x\rightarrow c}\dfrac{f'(x)-t_n'(x)}{n(x-c)^{n-1}}=\dots=\limlim{x\rightarrow c}\dfrac{f^{n}(x)-t_n^{n}(x)}{n!(x-c)^0}=\dfrac{f^{n}(c)-t_n^{(n)}(c)}{n!}=0$
	
\subsection{Beispiele}
	\subExBegin{a)}
		\item $f=a_kx^k+a_{k-1}x^{k-1}+\dots+a_0$ Polynom\\
			$c=0$\\
			$t_0\lrr{x}=a_0, t_1(x)=a_1x+a_0,\dots, t_k(x)=f(x)$\\
			$t_n(x)=f(x), n\geq k$
		\item $f(x)=e^x, c=0$\\
			$t_n(x)=1+x+\frac{x^2}{2}+\frac{x^3}{3!}+\dots+\frac{x^n}{n!}$\\
			(erste $n+1$ Glieder der Potenzreihe von $e^x$)
	\subExEnd
	
\subsection{Satz von Taylor}
	$I$ Intervall, $f:I\rightarrow\mr$, $\lrr{n+1}$-mal diffbar, $c\in I$\\
	$t_n$ $n$-tes Taylorpolynom von $f$ an $c$.\\
	Dann gibt es zu jedem $x\in I$ eine Zwischenstelle $y$ zwischen $x$ und $c$ ($y$ hängt von $x$ ab!) mit\\
	$R(x)=f(x)-t_n(x)=\frac{f^{(n+1)}(y)}{(n+1)!}\lrr{x-c}^{n+1}$\\
	das heißt $f(x)=\limsum{k=0}{n}\frac{f^{(k)}(c)}{k!}\lrr{x-c}^k+\frac{f^{(n+1)}(y)}{(n+1)!}\lrr{x-c}^{n+1}$ (\textbf{Taylorentwicklung} von $f$ an $c$)
	
	\textbf{Beweis}
	
	Sei $g(x) = \lrr{x-c}^{n+1}$\\
	$\frac{R(x)}{g(x)}=\frac{R(x)-R(c)}{g(x)-g(c)}\ouset{=}{\smt{2.MWS}}{}\underbrace{\frac{R'(x_1)}{g'(x_1)}}_{\scriptstyle x<x_1<c}=\frac{R'(x_1)-R'(c)}{g'(x_1)-g'(c)}\ouset{=}{\smt{2.MWS}}{}\underbrace{\frac{R''(x_2)}{g''(x_2}}_{\scriptstyle x_1<x_2<c}=\frac{R^{(n+1)}\lrr{x_{n+1}}}{g^{(n+1)}\lrr{x_{n+1}}}\ouset{=}{}{\scriptstyle R(x)=f(x)-t_n(x)}\underbrace{\frac{f^{(n+1)}\lrr{x_{n+1}}}{\lrr{n+1}!}}$
	
\subsection{Korollar}
	$f:I\rightarrow\mr$, $\lrr{n+1}$-mal diffbar.\\
	Ist $f^{(n+1)}=0$, so ist $f$ Polynom vom $\grad\leq n$\\
	(Folgt aus 12.4)
	
\subsection{Beispiel}
	Wie groß muss man $n$ wählen damit $\lrabs{\sin\lrr{x}-t_n(x)}<\frac{1}{100}$ für alle $x\in\lra{0,\frac{\pi}{2}}$ ($t_n$ $n$-te Taylorpolynom von $\sin$, $c=0$)\\
	$\lrabs{\sin\lrr{x}-t_n\lrr{x}}\ouset{=}{}{\smt{12.4}}\lrabs{\frac{\sin^{(n+1)}(y)}{(n+1)!}x^{n+1}}\leq\ouset{\mbox{max}}{}{\ouset{\scriptstyle y\in\lra{0,\frac{\pi}{2}}}{}{\scriptstyle x\in\lra{0,\frac{\pi}{2}}}}\lrabs{\frac{\sin^{(n+1)}(y)}{(n+1)!}x^{n+1}}=\frac{1}{\lrr{n+1}!}\lrr{\frac{\pi}{2}}^{n+1}<\frac{2^{n+1}}{\lrr{n+1!}}\ouset{<}{!}{}\frac{1}{100}$\\
	$n=9:\frac{2^{10}}{10!}\approx 0,003$
	
\subsection{Bemerkung}
	$f$ $2$-mal stetig differenzierbar. (12.4)
	(*) $f(x)=f(c)+f'(c)(x-c)+\frac{f''(y)}{2}\lrr{x-c}^2$ für ein $y$ zwischen $x$ und $c$\\
	Angenommen $f'(c)=0$, $f''(c)<0$\\
	$f''$ stetig $\Rightarrow\exists\delta: f''(x)<0\forall x\in\lra{c-\delta,c+\delta}$\\
	Für alle solchen $x$ ist $f(x)=f(c)+\frac{f''(y)}{2}\lrr{x-c}^2$\\
	$f(x)\leq f(c)\forall x\in\lra{c-\delta,c+\delta}$\\
	$c$ ist lokale Maximalstelle\\
	$f''(c)>0\Rightarrow c$ lokales Minimum
	
\subsection{Definition}
	$I$ Intervall, $f:I\rightarrow\mr$ unendlich oft differenzierbar, $c\in I$\\
	Dann heißt die Potenzreihe
	$$\suminf{k=0}\frac{f^{(k)}(x)}{k!}\lrr{x-c}^k$$
	\textbf{Taylorreihe} von $f$ um $c$