\newpage
\section{Determinanten}
	(Ausführliche Darstellung in WHK, Kapitel 10.4)
	
	Sei $K$ ein Körper, $n\in\mn$.\\
	Def. Abbildung $\det:M_n\lrr{K}\rightarrow K$ mit Eigenschaft:\\
	$A\in M_n\lrr{K}$ ist invertertierbar $\Leftrightarrow\det\lrr{A}\neq 0$\\
	Motivation: $n=1\qiad A=\lrr{a}$ invertiert $\Leftrightarrow a\neq 0\quad\det\lrr{\lrr{a}}:=a$\\
	$n=2. A=\lrv{a_{11}&a_{12}\\a_{21}&a_{22}}$\\
	Angenommen $A$ enthält keine Nullzeile, keine Nullspalte\\
	$A$ nicht-invertierbar $\Leftrightarrow$ Spalten von $A$ sind linear abhängig $\Leftrightarrow\lrv{a_{11}\\a_{21}}=c\cdot\lrv{a_{12}\\a_{22}}$\\
	$\Leftrightarrow a_{11}=c\cdot a_{12}, a_{21}=c\cdot a_{22}$ (wobei $a_{22},a_{12}\neq 0$, da $A$ keine Nullzeilen enthält)\\
	$\Leftrightarrow a_{11}a_{12}^{-1}=a_{21}a_{22}^{-1}\quad \lrr{=c}$\\
	$\Leftrightarrow a_{11}a_{22}=a_{21}a_{12}\Leftrightarrow a_{11}a_{22}-a_{12}a_{21}=0$\\
	Das stimmt auch, wenn $A$ Nullzeile oder Nullspalte enthält.\\
	Definiere $\det\lrr{A}=a_{11}a_{22}-a_{12}a_{21}$
	
\subsection{Definition}
	Sei $A\in M_n\lrr{k}$, $i,j\in\lrc{1,\dots,n}$\\
	$A_{ij}\in M_{n-1}\lrr{k}$ ist die Matrix, die aus $A$ durch Streichen der $i$-ten Zeile und $j$.ten Spalte entsteht.
	
	\textbf{Beispiel}
	
	$A=\lrv{1&2&3\\4&5&6\\7&8&9}\quad A_{11}=\lrv{5&6\\8&9}\quad A_{23}=\lrv{1&2\\7&8}$

\subsection{Definition}
	$A\in M_n\lrr{k}$\\
	Ist $n=1$, das heist $A=\lrr{a}$, so $\det\lrr{A}=a$.\\
	Definiere für $n\geq 1 \;\; \det\lrr{A}$ rekursiv durch\\
	$\det\lrr{A}=\limsum{j=1}{n}\lrr{-1}^{j+1}a_{ij}\cdot\det\lrr{A_{ij}}=a_{11}\det\lrr{A_{11}}-a_{12}\det\lrr{A_{12}}+\dots\pm a_{1n}\dot\lrr{A_{1n}}$\\
	$\det\lrr{A}$ heißt \textbf{Determinante} von $A$
	
	$\lra{n=2:\quad \det\lrr{A}=a_{11}\det\lrr{A_{11}}-a_{12}\det\lrr{A_{12}}=a_{11}\cdot a_{22}-a_{12}a_{21}}$
	
	\textbf{Entwicklung nach der ersten Zeile}
	
\subsection{Beispiel}
	\subExBegin{a)}
		\item $\det\lrv{a_{11}&a_{12}&a_{13}\\a_{21}&a_{22}&a_{23}\\a_{31}&a_{32}&a_{33}}=a_{11}\det\lrv{a_{22}&a_{23\\a_{32}&a_{33}}-a_{12}\det\lrv{a_21&a_{23}\\a_{31}&a_{33}}+a{13}\det\lrv{a_{21}&a_{22}\\a_{31}&a_{32}}\\
			=a_{11}a_{22}a_{33}-a_{11}a_{23}a_{32}-a_{12}a_{21}a_{33}+a_{12}a_{23}a_{31}+a_{13}a_{32}a_{21}-a_{13}a_{31}a_{22}$
			
			\textbf{Regel von Sarrus}
			
			%TODO - Graphik
			
			\begin{tikzpicture}
				\draw (0,0)node{$\lrv{a_{11}&a_{12}&a_{13}&a_{11}&a_{12}\\a_{21}&a_{22}&a_{23}&a_{21}&a_{22}\\a_{31}&a_{32}&a_{33}&a_{31}&a_{32}$}
			\end{tikzpicture}
			\backslash\backslash\backslash $-///$
			
			Entsprechende Formeln gibt es für alle $n$. Sie enthalten $n!$ Summanden von $n$ Faktoren, versehen mit $\pm$.\\
			\textbf{Leibniz'sche Determinantenformeln} (WHK Satz 10.43)
		\item $\det\lrv{a_{11}&0&\dots&0\\\vdots&a_{22}&0&0\\a_{n_1}&\dots&\dots&a_{nn}}=a_{11}\cdot a_{22}\cdot\dots\cdot a_{nn}$\\
			Oberhalb der Diagonalen stehen Nullen - heist untere Dreiecksmatrix\\
			Richtig für $n=1$\\
			Allgemeiner Fall durch Induktion $n-1\rightarrow n$\\
			$\det\lrv{a_{11}&&0\\\vdots&\vdots&\vdots\\*&\dots&a_{nn}}\ouset{=}{}{\smt{6.2}}=a_{11}\det