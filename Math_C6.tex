\newpage
\section{Determinanten}
	(Ausführliche Darstellung in WHK, Kapitel 10.4)

	Sei $K$ ein Körper, $n\in\mn$.\\
	Def. Abbildung $\det:M_n\lrr{K}\rightarrow K$ mit Eigenschaft:\\
	$A\in M_n\lrr{K}$ ist invertertierbar $\Leftrightarrow\det\lrr{A}\neq 0$\\
	Motivation: $n=1\quad A=\lrr{a}$ invertiert $\Leftrightarrow a\neq 0\quad\det\lrr{\lrr{a}}:=a$\\
	$n=2. A=\lrv{a_{11}&a_{12}\\a_{21}&a_{22}}$\\
	Angenommen $A$ enthält keine Nullzeile, keine Nullspalte\\
	$A$ nicht-invertierbar $\Leftrightarrow$ Spalten von $A$ sind linear abhängig $\Leftrightarrow\lrv{a_{11}\\a_{21}}=c\cdot\lrv{a_{12}\\a_{22}}$\\
	$\Leftrightarrow a_{11}=c\cdot a_{12}, a_{21}=c\cdot a_{22}$ (wobei $a_{22},a_{12}\neq 0$, da $A$ keine Nullzeilen enthält)\\
	$\Leftrightarrow a_{11}a_{12}^{-1}=a_{21}a_{22}^{-1}\quad \lrr{=c}$\\
	$\Leftrightarrow a_{11}a_{22}=a_{21}a_{12}\Leftrightarrow a_{11}a_{22}-a_{12}a_{21}=0$\\
	Das stimmt auch, wenn $A$ Nullzeile oder Nullspalte enthält.\\
	Definiere $\det\lrr{A}=a_{11}a_{22}-a_{12}a_{21}$

\subsection{Definition}
	Sei $A\in M_n\lrr{k}$, $i,j\in\lrc{1,\dots,n}$\\
	$A_{ij}\in M_{n-1}\lrr{k}$ ist die Matrix, die aus $A$ durch Streichen der $i$-ten Zeile und $j$.ten Spalte entsteht.

	\textbf{Beispiel}

	$A=\lrv{1&2&3\\4&5&6\\7&8&9}\quad A_{11}=\lrv{5&6\\8&9}\quad A_{23}=\lrv{1&2\\7&8}$

\subsection{Definition: Determinante}
	$A\in M_n\lrr{k}$\\
	Ist $n=1$, das heist $A=\lrr{a}$, so $\det\lrr{A}=a$.\\
	Definiere für $n\geq 1 \;\; \det\lrr{A}$ rekursiv durch\\
  $\det\lrr{A}=\limsum{j=1}{n}\lrr{-1}^{j+1}a_{ij}\cdot\det\lrr{A_{ij}}=a_{11}\det\lrr{A_{11}}-a_{12}\det\lrr{A_{12}}+\dots\pm
  a_{1n}\cdot\lrr{A_{1n}}$\\
	$\det\lrr{A}$ heißt \textbf{Determinante} von $A$

	$\lra{n=2:\quad \det\lrr{A}=a_{11}\det\lrr{A_{11}}-a_{12}\det\lrr{A_{12}}=a_{11}\cdot a_{22}-a_{12}a_{21}}$

	\textbf{Entwicklung nach der ersten Zeile}

\subsection{Beispiel}
	\subExBegin{a)}
		\item $\det\lrv{a_{11}&a_{12}&a_{13}\\a_{21}&a_{22}&a_{23}\\a_{31}&a_{32}&a_{33}}=a_{11}\det\lrv{a_{22}&a_{23}\\a_{32}&a_{33}}-a_{12}\det\lrv{a_21&a_{23}\\a_{31}&a_{33}}+a{13}\det\lrv{a_{21}&a_{22}\\a_{31}&a_{32}}\\
			=a_{11}a_{22}a_{33}-a_{11}a_{23}a_{32}-a_{12}a_{21}a_{33}+a_{12}a_{23}a_{31}+a_{13}a_{32}a_{21}-a_{13}a_{31}a_{22}$

			\textbf{Regel von Sarrus}

			%TODO - Graphik

			%\begin{tikzpicture}
      %  \draw (0,0) node {$\lrv{a_{11}&a_{12}&a_{13}&a_{11}&a_{12}\\a_{21}&a_{22}&a_{23}&a_{21}&a_{22}\\a_{31}&a_{32}&a_{33}&a_{31}&a_{32}}$}
			%\end{tikzpicture}
			%\backslash\backslash\backslash $-///$

			Entsprechende Formeln gibt es für alle $n$. Sie enthalten $n!$ Summanden von $n$ Faktoren, versehen mit $\pm$.\\
			\textbf{Leibniz'sche Determinantenformeln} (WHK Satz 10.43)
		\item $\det\lrv{a_{11}&0&\dots&0\\\vdots&a_{22}&0&0\\a_{n_1}&\dots&\dots&a_{nn}}=a_{11}\cdot a_{22}\cdot\dots\cdot a_{nn}$\\
			Oberhalb der Diagonalen stehen Nullen - heist untere Dreiecksmatrix\\
			Richtig für $n=1$\\
			Allgemeiner Fall durch Induktion $n-1\rightarrow n$\\
			%TODO Abschreiben...
      %$\det\lrv{a_{11}&&0\\\vdots&\vdots&\vdots\\*&\dots&a_{nn}}\ouset{=}{}{\smt{6.2}}=a_{11}\det
  \subExEnd

\subsection{Laplace'scher Entwicklungssatz}
  \subExBegin{a)}
    \item (Entwicklungssatz nach der $i$- ten Zeile)
      Sei $i\in\lrc{1,...,n}.$\\
      Dann $\det(A)=\limsum{j=1}{n}(-1)^{i+j}a_{ij}\det(A_{ij})$
    \item (Entwicklungssatz nach der $j$- ten Spalte)
      Sei $j\in\lrc{1,...,n}$.\\
      Dann $\det(A)=\limsum{i=1}{n}(-1)^{i+j}a_{ij}\det(A_{ij})$
  \subExEnd
  Struktur:\begin{tabular}{cccc}+&-&+&-\\-&+&-&+\\+&-&+&-\\-&+&-&+\end{tabular}

\subsection{Korollar}
  \[\det(A)=\det\lrr{A^t}\]

\subsection{Beispiel}
  \begin{enumerate}[a)]
    \item $A=\begin{pmatrix}2&0&3\\-1&0&4\\2&3&-1\end{pmatrix}\in M_3(\mq)$\\
      Definition von $\det(A)$:\\
      $\det(A)=2\cdot\lrv{0&4\\3&-1}+e\cdot\det\lrv{-1&0\\2&3}=-2\cdot
      12-3\cdot 3=-33$.\\ Entwicklung nach 2. Spalte:
      $\det(A)=(-3)\det\lrv{2&3\\-1&4}=(-3)\cdot 11=-33$.

      Allgemeine Regel: Berechne Determinante durch Entwicklung der Zeile oder
      Spalte mit möglichst vielen Nullen.

    \item
      $\det\lrv{a_{11}&\dots&a_{1n}\\0&\ddots&\vdots\\0&0&a_{nn}}\ouset{=}{6.5}{}\det\lrv{a_{11}&0&0\\\vdots&\ddots&0\\a_{1n}&\dots&a_{nn}}=a_{11}\cdot...\cdot a_{nn}$\\
      (Determinanten oberer Dreieckmatrix)
  \end{enumerate}
\subsection{Rechenregeln für Determinanten}
  $A,B\in M_n(K),s_1,...,s_n$ Spaltenvektoren von $A$, d.h. $A=(s_1,s_2,...,s_n)$
  \begin{enumerate}[a)]
    \item Ist $t_i\in K^n$, $A'=(s_1,...,s_{t-1},t_i,s_{i+1},...,s_n)$, so gilt
      $\det(s_1,...,s_{i-1},s_i+t_i,s_{i+1},s_n)=\det(A)+\det(A')$
    \item $\det(s_1,...,s_{i-1},a\cdot s_i,s_{i+1},...,s_n)=a\cdot\det(A)$
  \end{enumerate}
  a) und b) besagen: Die Determinante ist eine lineare Abbildung bezüglich
  jeder Spalte, wenn die übrigen Spalten festgehalten werden
  \begin{enumerate}[c)]
    \item $\det(a\cdot A)=a^n\cdot\det(A)$
    \item Vertauschung zweier Spalten von $A$ ändert das Vorzeichen der
      Determinante.
    \item Sind zwei Spalten von $A$ gleich, so ist $\det(A)=0$.
    \item $\det(s_1,...,s_{i-1},s_i+a\cdot s_j,s_{i+n},...,s_n)=\det(A)$ für
      alle $i\neq j,a\in K$
    \item Determinantenmultiplikationssatz: $\det(A\cdot
      B)=\det(A)\cdot\det(B)$
  \end{enumerate}

  Aussagen von a), b), d), e), f) gelten analog für Zeilen anstelle von
  Spalten.

  \textbf{Beachte:} Im Allgemeinen ist $\det(A+B)\neq\det(A)+\det(B)$\\

  $E_2\quad 2\times 2$- Einheitsmatrix.\\
  $\det(E_2)+\det(E_2)=2\neq\det(E_2+E_2)=\det\lrv{2&0\\0&2}=4$

\subsection{Bemerkung: Determinatenbestimmung}
  \begin{enumerate}[a)]
    \item Elementare Zeilenumformungen (ebenso wie elementare
      Spaltenumformungen) haben nach 6.7 folgenden Effekt auf Determinanten:
      \begin{itemize}[-]
        \item Addition der Vielfachen einer Zeile zu einer anderen Zeile:\\
          $\rightarrow\det(A)$ ändert sich nicht.
        \item Vertauschung zweier Zeilen:\\
          $\rightarrow\det(A)$ ändert Vorzeichen
        \item Multipilikation einr Zeile mit $a\in K$:\\
          $\rightarrow$ Determinante ändert sich um Faktor $a$.
      \end{itemize}
    \item Strategie zu Determinantenberechnung:\\
      Verwende elementarer Zeilen-/Spaltenumformungen von erstem Typ von
      Spalte/Zeile zu erhalten, wo alle Einträge bis auf höchstens einen 0
      sind.\\
      Entwickle nach dieser Spalte/Zeile.\\
      Oder: Erzeuge wie bei Gauß- Verfahren durch elementare Zeilenumformungen
      der ersten beiden Typen und gegebenfalls Spaltenvertauschungen
      (Vorzeichenwechsel beachten!) eine Dreiecksmatrix.
      Determinantenberechnung trivial.
  \end{enumerate}

\subsection{Beispiel}
  $A=\begin{pmatrix}0&1&2&3\\-1&0&0&4\\1&3&4&2\\0&3&0&1\end{pmatrix}, k=\mq,\mr$\\
  $\det(a)=-\det\lrv{-1&0&0&4\\0&1&2&3\\1&3&4&2\\0&3&0&1}=
  -\det\lrv{-1&0&0&4\\0&1&2&3\\0&3&4&6\\0&3&0&1}=
  -\det\lrv{-1&0&0&4\\0&1&2&3\\0&0&-2&-3\\0&0&-6&-8}
  =-\det\lrv{-1&0&0&4\\0&1&2&3\\0&0&-2&-3\\0&0&0&1}=-2$.

\subsection{Satz: Invertierbarkeit von Matrizen}
  $A\in M_n(K)$. Dann sind äquivalent:
  \begin{enumerate}[(1)]
    \item $A$ ist invertierbar
    \item $\det(A)\neq 0$
    \item Die Spalten von $A$ sind linear unabhängig
    \item die Zeilen von $A$ sind linear unabhängig
  \end{enumerate}
  Ist $\det(A)\neq 0$, so ist $\det\lrr{\ifu{A}}=\ifu{\det(A)}\lrr{=\frac{1}{\det(A)}}$\\
  \textbf{Beweis:}
  \begin{itemize}
    \item (1)$\Rightarrow$(2):\\
      $A\cdot \ifu{A}=E_n$. $1=\det(E_n)=\det(A\cdot\ifu{A})\ouset{=}{}{6.7g)}\det(A)\cdot\det\lrr{\ifu{A}}\\
      \Rightarrow\det(A)=0$
    \item (2)$\Rightarrow$(3):\\
      Seien $s_1,...,s_n$ Spalten von $A$. Angenommen, die Spalten sind linear
      Abhängig. Das heißt $\exists i: S_i=\limsum{j\neq i}{}a_js_j$ mit
      geeignetem $a_j\in K$.\\
      $\det(A)=\det(s_1,...,s_n)\ouset{=}{}{6.7f}\det(s_1,...,s_i-\limsum{j\neq
      i}{}a_js_j,s_{i+1},...,s_n)=$\\
      $\det(s_1,...,s_{i-j},0,s_{i+1},...,s_n)\ouset{=}{}{6.7b}0$\lightning
    \item (3)$\Rightarrow$(4):\\
      4.18: Zeilenrang $=$ Spaltenrang
    \item (4)$\Rightarrow$(1):\\
      5.11
  \end{itemize}

\subsection{Korollar}
Sei $A\in M_n(K)$, $b\in K^n$. Ist $\det(A)\neq 0$, so hat das LGS $A\cdot x=b$
genau eines Lösung, nämlich $x=\ifu{A}\cdot b$.

\subsection{Definition: Adjunkte}
$A\in M_n(K)$. Die \textbf{Adjunkte} $A^{ad}$ zu $A$ ist $n\times n$- Matrix
$A^{ad}=(b_{ij})_{i,j\neq 1}=1,...,n$ mit $b_{ij}=(-1)^{i+j}\det(A_{ji})$
(Indizes beachten!)

\subsection{Satz}
$A\in M_n(K)$
\begin{enumerate}[a)]
  \item $A^{ad}\cdot A=A\cdot A^{ad}=(\det(A))E_n$
  \item Ist $\det(A\neq 0)$, so ist $\ifu{A}=\ifu{(\det(A))}\cdot A^{ad}$
\end{enumerate}
(Beweis mit 6.4 und 6.7)

\subsection{Beispiel}
$A=\lrv{a_{11}&a_{12}\\a_{21}&a_{22}}$, $\det(A)=a_{11}a_{22}-a_{21}a_{21}\neq 0$\\
$\ifu{A}=\frac{1}{a_{11}a_{22}-a_{21}a_{12}}\lrv{a_{22}&-a_{12}\\-a_{21}a_{11}}$

\subsection{Bemerkung}
$\alpha:V\rightarrow V$ lineare Abbildung, $\mathcal{B},\mathcal{B}'$ Basen von $V$.\\
$A_\alpha^{\mathcal{B}'}\cdot S$, $S=S_{\mathcal{B}},\mathcal{B}'$ (5.20)\\
$\det\lrr{A_\alpha^{\mathcal{B}'}}=\det\lrr{\ifu{S}\cdot A_\alpha^{\mathcal{B}'}\cdot
S}\ouset{=}{}{6.7g}=\det\lrr{\ifu{S}}\cdot\det\lrr{A_\alpha^{\mathcal{B}}}\cdot\det(S)\ouset{=}{}{6.10}\ifu{\det(S)}\cdot\det(S)\cdot\det\lrr{A_\alpha^{\mathcal{B}}i}=\det\lrr{A_\alpha^{\mathcal{B}}}$

\textbf{Def.:} $\det(\alpha):=\det\lrr{A_\alpha^{\mathcal{B}}}$ unabhängig von
der Wahl von $\mathcal{B}$.
